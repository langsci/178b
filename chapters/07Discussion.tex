% !TEX root = ../main.tex
\chapter[Discussion]{Discussion: Multi-dimensionality of linguistic forms}\label{Discussion}


%%% 属性叙述 vs. 事象叙述

%%% 生成文法の、音声部門、統語部門、意味部門の独立性仮説に対する反証?

%%% topic-comment構造と主語-述語構造の類似 (, 1990: p.280ff.)
%%% Whitman (1986)のまとめ参照

%%----------------------------------------------------
\section{Summary of findings}

The findings so far are summarized in Table \ref{TopSummary} and \ref{FocSummary}.

\begin{table}[hbt]
	\caption{Summary of topic}
	\label{TopSummary}
	\begin{center}
	\begin{tabular}{|l|l|c|c|c|}
	\hhline{-----}
	Activation status & Given-new taxonomy & Particles & Word order & Intonation \\
	\hhline{|-|-|-|-|-|}
	 \multirow{4}{*}{Active} & \multirow{3}{*}{Strongly evoked} & \multicolumn{3}{c|}{(Zero pronoun)} \\
	\hhline{|~|~|-|-|-|}
	  & & \multirow{3}{*}{\ci{toiuno-wa, wa}, {\O}} & {Post-predicate} & \multirow{2}{*}{Clausal IU} \\
	\hhline{|~|~|~|-|~|}
	  &                &  & \multirow{5}{*}{Clause-initial} &  \\
	\hhline{|~|-|~|~|-|}
	  & Evoked          &  &  & \multirow{4}{*}{Phrasal IU} \\
	\hhline{|-|-|-|~|~|}
	 \multirow{2}{*}{Semi-active} & Inferable & \ci{wa}, {\O} &  &  \\
	\hhline{|~|-|-|~|~|}
	  & Declining & \multirow{2}{*}{\ab{cop}-\ci{kedo/ga}, {\O}}  &   &  \\
	\hhline{|-|-|~|~|~|}
	 \multirow{2}{*}{Inactive} & Unused  &  &   &  \\
	\hhline{|~|-|-|-|-|}
	  & Brand-new &  --  & -- & -- \\
	\hhline{-----}
	\end{tabular}\\
	\end{center}
\end{table}


\begin{table}[hbt]
\centering
 \caption{Summary of (broad) focus}
 \label{FocSummary}
 \begin{tabular}{|l|c|c|c|}
 \hhline{|-|-|-|-|}
            & Particles  & Word order    & Intonation \\
 \hhline{|-|-|-|-|}
  A         & \ci{ga}    & \multirow{4}{*}{Pre-predicate} & \multirow{4}{*} {Clausal IU} \\
 \hhline{|-|-|~|~|}
  Agent S   & \ci{ga}    &  &  \\
\hhline{|-|-|~|~|}
  Patient S & \ci{ga, \O} &  &  \\
 \hhline{|-|-|~|~|}
  P         & \ci{\O}         &  &  \\
 \hhline{|-|-|-|-|}
 \end{tabular}
\end{table}

%where \ci{N} and \ci{V} mean noun and verb,
%the subscripts $T$ and $F$ mean topic and focus, respectively,
%* means repetition of linguistic expression more than 0 times,
%+ means repetition more than 1 times, and
%{\iub} indicates intonational-phrase boundary.
%\ci{TC} and \ci{AC} mean topic and focus codings, respectively.
%%
%\ex.\label{Dis:Ex:PredFocus}Predicate-focus
% \a. Unestablished topic: [\EM{N}(-TC)]$_{T}$ {\iub} [({N}(-AC))* \EM{V}]$_{F}$
% \b. Established topic: [(N(-TC))*]$_{T}$ [({N}(-AC))* \EM{V}]$_{F}$ [({N}(-TC))*]$_{T}$
%
%\ex.\label{Dis:Ex:ArgFocus}Argument focus
% \a. {[\EM{N-AM}]$_{F}$}$^{+}$ [(V)]$_{T}$

Overall,
I showed that correlated but distinct features affect the choice of particles, word order, and intonation in spoken Japanese.
The features proposed are summarized in \ref{ISFeatures} in Chapter \ref{Framework},
which is repeated here as \Next for convenience.
%
\ex.\label{Dis:Ex:ISFeatures}
\begin{tabular}{lll}
	 & topic & focus \\
	a. & presupposed & asserted \\
	b. & evoked & brand-new \\
	c. & definite & indefinite \\
	d. & specific & non-specific \\
	e. & animate & inanimate \\
	f. & agent & patient \\
	g. & inferable & non-inferable \\
%	h. & entity & proposition \\
\end{tabular}


In Chapter \ref{Particles},
I concentrated on particles.
Topic markers such as \ci{toiuno-wa}, \ci{wa}, and \ci{kedo/ga}
are sensitive to the assumed statuses in the given-new taxonomy of the referent in question.
All topic markers code elements that are presupposed to be
shared between the speaker and the hearer
and cannot be negated in a normal way.
Namely, topic markers are sensitive to
a and b in \ref{Dis:Ex:ISFeatures}.
The marker \ci{toiuno-wa} codes elements referring to an entity in evoked status in the hearer's mind.
The marker \ci{wa} codes elements referring to an entity in inferable status,
in addition to elements that can be coded by \ci{toiuno-wa}.
The marker \ci{kedo/ga} preceded by the copula \ci{da} or \ci{desu}
codes elements referring to an entity that is declining or unused in the assumed hearer's mind.
Topic markers are optional except for contrastive topics.
In a formal speech style,
topic markers tend to appear.
In addition to whether the referent in question is evoked or not,
I also showed that the topic markers are partially sensitive to grammatical function (f in \ref{Dis:Ex:ISFeatures});
when the clause has two evoked arguments, A and P,
A is more likely to be coded by topic markers (in this case, \ci{wa}),
rather than P.

Case markers are, on the other hand, sensitive to whether the referent is (part of) an assertion or not (a in \ref{Dis:Ex:ISFeatures}),
in addition to grammatical functions (f in \ref{Dis:Ex:ISFeatures}).
A, agent S, and optionally patient S are coded by \ci{ga},
whereas patient S and P tend to be coded by \ci{\O}.
A, S, and P in the argument focus or narrow focus environment are coded by explicit markers.
I (and the previous literature) also suggested the possibility that
\ci{ga} and \ci{o} are sensitive to animacy (e in \ref{Dis:Ex:ISFeatures}).

In Chapter \ref{WordOrder},
I focused on word order.
I showed that shared elements, which correlate with topics, tend to appear clause-initially irrespective of the status of the given-new taxonomy.
Strongly evoked elements can appear post-predicatively
especially in conversation.
Post-predicate elements are sensitive to the given-new taxonomy (b in \ref{Dis:Ex:ISFeatures}),
while clause-initial elements are sensitive to identifiability.
%which appear to be related to definiteness (\ref{Dis:Ex:ISFeatures}c).
On the other hand, foci tend to appear pre-predicatively
(i.e., immediately before the predicate).
Pre-predicate elements tend to refer to non-shared entities,
in contrast with clause-initial topics.
Word order is also sensitive to grammatical function (f in \ref{Dis:Ex:ISFeatures}),
as classically observed.
The referent of clause-initial elements is referred to by zero pronouns in the following discourse,
while the referent of pre-predicate elements repeatedly appears as full NPs.
%When established topics appear explicitly,
%they are produced clause-initially or post-predicatively in a reduced form (intonationally and phonetically).
%Finally, 
%Pre-predicate elements form a focus with the predicate.
%Word order is also sensitive to grammatical function (\ref{Dis:Ex:ISFeatures}f) as classically observed.
In terms of word order,
I proposed that three inter-related principles, repeated here as \ref{Disc:oldnewprinciple}, \ref{Disc:IScontinuityP}, and \ref{Disc:PerFirstPrinciple} are working to determine word order of spoken Japanese.
Principles \ref{Disc:oldnewprinciple} and \ref{Disc:PerFirstPrinciple}
predict that topics appear clause-initially,
while Principle \ref{Disc:IScontinuityP} and the assumption that Japanese is a verb-final language predict that
the focus appears pre-predicatively.
%
\ex. \label{Disc:oldnewprinciple}\tl{From-old-to-new principle}:
 In languages in which word order is relatively free,
 the unmarked word order of constituents is old,
 predictable information first and new, unpredictable information last.
 \hfill{(\citeA[][54]{kuno78}, \citeA[][p.\ 326]{kuno04})}

\ex. \label{Disc:IScontinuityP}\tl{Information-structure continuity principle}:
 A unit of information structure must be continuous in a clause;
 i.e., elements which belong to the same unit are adjacent to each other.

\ex. \label{Disc:PerFirstPrinciple}\tl{Persistent-element-first principle}:
 In languages in which word order is relatively free,
 the unmarked word order of constituents is persistent element first and non-persistent element last.

Perhaps, there is no principle that predicts the order of strongly evoked elements because they are not necessary;
the hearer is assumed to be able to identify the referent
because it is strongly evoked.
They are produced for some intonational or interactional reasons
as has been discussed in \ref{WO:PostP:Motivations}.

In Chapter \ref{Intonation},
I discussed intonation.
I showed that
evoked, inferable, declining, and unused topics tend to be produced in an intonation unit separately from the predicate,
while strongly evoked topics tend to be produced in an intonation unit together with the predicate.
On the other hand,
the broad focus tends to appear in an intonation unit with the predicate
to form a unit of predicate focus structure.
I proposed two principles determining intonation units in Japanese,
repeated here as \ref{Disc:IUIconicP} and \ref{Disc:IUActCostP}.
Principle \ref{Disc:IUIconicP} predicts that
a topic appears in an intonation contour and a focus appears in another intonation contour,
whereas Principle \ref{Disc:IUActCostP} predicts that
strongly evoked topics are glued to an IU of focus.
%
\ex.\label{Disc:IUIconicP}\tl{Iconic principle of intonation unit and information structure}:
	In spoken language,
	an IU tends to correspond to a unit of information structure.

\ex.\label{Disc:IUActCostP}\tl{Principle of intonation unit and activation cost}:
     all substantive IUs have similar activation costs;
     there are few IUs with only a strongly evoked element or
     those with too many new elements.

To be more precise, these principles predict that
when the activation cost of a topic is high,
it is separated intonationally from the focus predicate, as in \Next[a];
whereas when the activation cost of a topic is low,
it is produced with the focus predicate, as in \Next[b-c].
A box corresponds to an IU.
%
\ex.
 \a. \fbox{Topic} \fbox{Focus}
 \b. \fbox{Topic Focus}
 \b. \fbox{Focus Topic}

%%----------------------------------------------------
\section{Competing motivations}\label{Disc:CompMotivations}

As summarized above,
there is no single feature (such as topic or focus) which determines
the choice of particles, word order, and intonation;
multiple features influence a single linguistic expression.
This is not a rare phenomenon;
rather, it is frequently observed in languages and
it is a source of language change.
\citeA{comrie79} called this variability ``seepage''.
As has been discussed in \S \ref{Par:CasePar:Ga:GaAnim},
\ci{ko} in Hindi codes definite or animate direct object;
there is no single feature that determines the use of \ci{ko}.
Citing \cite{poppe70}, he discusses another example from Mongolian.
According to Poppe, the accusative suffix \ci{-iig} only attaches to certain kinds of direct objects.
Human direct objects are always followed by the suffix as exemplified in
\Next.
%
\ex.\label{Disc:Ex:Mongolian1}
 \ag. dor\v{z} bag\v{s}-\EM{iig} zalav \\
	dorj teacher-{\ab{do}} invited \\
	`Dorj invited the teacher.'
 \bg. bid nar olan x\"{u}n-\EM{iig} \"{u}zsen \\
	we ? many people-{\ab{do}} saw \\
	`We saw many people.'
	\hfill{\cite[18]{comrie79}}

On the other hand, non-human direct objects are optionally followed by the suffix, as in \Next.
In this case,
definiteness plays an important role.
To complicate things,
the suffix also attaches to indefinite direct objects
when they are apart from the verb.
%
\ex.\label{Disc:Ex:Mongolian2}
 \ag. \v{c}oidog zurag zurav \\
	Choidog picture painted \\
	`Choidog painted a picture'
 \bg. zurag-iig Choidog zurav \\
	picture-\ab{do} Choidog painted \\
	`Choidog painted the picture. (As for the picture, it was Choidog that painted it.)'
	\hfill{\cite[19]{comrie79}}

The distinction between the so-called accusative marker \ci{o} and the zero particles in Japanese is similar to the use (or non-use) of this suffix \ci{-iig} in Mongolian.
%in the sense that they are sensitive to multiple features.
The choice between \ci{o} and the zero particles is reported to be determined by definiteness, animacy, and word order.
Definite or animate objects are more likely to be coded by \ci{o}
rather than the zero particles \cite{minashima01,fry01,kurumadajaeger13,kurumadajaeger15}.
Also,
according to \citeA{tsutsui84,matsuda96}, and \citeA{fry01},
verb-adjacent objects are more likely to be zero-coded (hence less likely to be \ci{o}-coded),
while non-verb-adjacent objects are more likely to be coded by \ci{o},
although the distinction is subtle.


\citeA{dubois85} argues that multi-dimensionality of a linguistic expression is based on ``competing motivations''.
%In a situation where multiple (sometimes contradicting) motivations compete with each other,
An example of competing motivations that Du Bois provides and is relevant to this study is
that of the distinction between ergative-absolutive and nominative-accusative languages.
%
\begin{quote}
The reason that not all languages are ergative -- i.e.~that some languages choose the `option' of categorizing S with A rather than with O [P in terms of this study] -- 
is that there is another motivation which competes for the same limited good,
the structuring of the person-number-role paradigm. [...]
S and A are united by their tendency to code referents which are human,
(relatively) agentive, and maintained as topics over significant stretches of discourse (`thematic').
Thus, a discourse pressure to roughly mark topic/agent motivates nominative-accusative morphology,
while a discourse pressure to roughly mark new information motivates ergative-absolutive morphology.
These two pressures may be seen as competing to overlay a secondary function on the existing A/S/O base (though this formulation is of course somewhat oversimplified). [...]
Thus the answer to the question as to why not all languages are ergative is simply that,
while there is a strong discourse pressure which motivates an absolutive category, there is an equally strong -- possibly stronger --
discourse pressure which motivates a nominative category.
Both motivations cannot prevail in the competition for control of the linguistic substance of this paradigm.
\cite[354--355]{dubois85}
\end{quote}
%
My study showed competing motivations that affect
choices of particles, word order, and intonation in spoken Japanese.
For example,
as has been discussed in \S \ref{Par:Dis:Markedness} and \citeA{nakagawa13m},
case particles are sensitive to focushood and thus
P and patient S are unmarked (zero-coded).
On the other hand, topic markers are sensitive to topichood and thus
A and agent S are unmarked in another dialect, Kansai Japanese.

If a single feature ``topic'' or ``focus'' determines the choice of word order and particles,
it is expected, for example, that
all clause-initial elements are coded by topic markers
because both clause-initial elements and those coded by topic markers are topics.
However, this is not the case,
as shown in \S \ref{WO:ClauseInit:Ident:Topic}.
Although clause-initial elements tend to be coded by topic markers,
not all clause-initial elements are coded by topic markers.
This is because word order and topic coding are sensitive to different features, while both of them are sensitive to topichood and focushood; clause-initial elements are sensitive to identifiability,
whereas topic markers are sensitive to activation status of the referent in question.

The claim of this study is an elaboration of the claim made by
\citeA{lithompson76} that
Japanese is a subject-prominent and topic-prominent language.
In terms of this study,
the claim is elaborated in the following way;
Japanese is sensitive to various features related to topichood and focushood such as presupposition vs.~assertion (a in \ref{Dis:Ex:ISFeatures})
and the status of the given-new taxonomy (b in \ref{Dis:Ex:ISFeatures}),
in addition to grammatical function (f in \ref{Dis:Ex:ISFeatures}).

The theory of competing motivations and correlating features of topic and focus \ref{Dis:Ex:ISFeatures} predicts that
there are other types of languages such as animacy-prominent languages and specificity-prominent languages.
As far as I am aware,
there are at least what I call animacy-prominent languages according to the literature \cite[\EMt{inter alia}]{dahlfraurud96,minkoff00,deswartetal07}.
For example, in grammatical sentences in Mam-Maya,
the subject is as animate as, or more animate than the object \cite{minkoff00}.
%%% 要文献
%%% 例
Another well-known example is Navajo (Athapaskan).
In Navajo, the order of S and P can be either SP or PS.
In the case of an SP order, the marker \ci{yi} attaches before the verb;
in the case of a PS order, the marker \ci{bi} attaches to the verb \cite{hale72,frischberg72}.
This is exemplified in \Next.
In \Next[a],
where the subject `horse' precedes the object `mule',
the affix \ci{yi} attaches to the verb.
In \Next[b], on the other hand,
where the object precedes the subject,
\ci{bi} is used.
%
%%% 要文献
%%% 例
\ex.
 \ag. lh\'{\i}\'{\i} dzaan\'e\'ez \EM{yi}-ztalh \\
      horse mule him-kicked \\
      `The horse kicked the mule.' \hfill{(SP)}
 \bg. dzaan\'e\'ez lh\'{\i}\'{\i} \EM{bi}-ztalh \\
      mule horse him-kicked \\
      `The horse kicked the mule.' \hfill{(PS)}\\
 \b.[] \hfill{\cite[300]{hale72}}

When the subject and the object are equally animate, as in \Last,
both \ci{yi-} and \ci{bi-} constructions can be used.
However,
when the subject is more animate than the object,
only \ci{yi-}construction with the SP order is grammatical;
while when the object is more animate than the subject,
only \ci{bi-} construction with the PS order is grammatical.
These languages can be called animate-prominent languages
in the sense that
animacy constrains word order or grammatical functions.


%the topic in this study was defined mainly based on Japanese.
%there is no single marker that codes topics.
%in some languages, there might not be a clear linguistic category that codes topic.
%English, for example,
%does not seem to code topic.
%maybe the following case:
%A whale--it is a largest creature in the sea.

Finally, I point out that this kind of multivariate analysis
is not compatible with theories like generative grammar.
For example,
\citeA{endo14}, following Rizzi's cartography theory \cite[e.g.,][]{rizzi97,rizzi04},
points out that ``an information focus occurs immediate left to the verb'' (p.~170).%
 \footnote{
 An information focus is ``the answer to \ci{wh}-questions and the target of negation'' (ibid.),
 which is the same focus discussed in this study.
 }
This observation is compatible with that of \citeA{kuno78}.
In the following example \Next[A],
\ci{hon} `book' is a focus because it is the answer to the \ci{wh}-question \Next[Q].
The focus appears immediately before the verb \ci{kai-masi-ta} `bought'.
%
\ex.
 \a.[Q:] What did you buy?
 \bg.[A:] watasi-wa \EM{hon-o} kai-masi-ta \\
          \ab{1}\ab{sg}-\ab{top} book-\ab{acc} buy-\ab{plt}-\ab{past} \\
          `I bought a book.'
  \hfill{\cite[170--171]{endo14}}

As we immediately notice, however,
the focus \ci{hon} `book' is the object (P) of the sentence at the same time.
In the cartography framework,
it is not clear how to represent an element which is a focus and the object
at the same time.



%%----------------------------------------------------
%\section{Unit of processing and noun incorporation}

%%----------------------------------------------------
%\section{``A sentence'' as a unit of information structure}

%%----------------------------------------------------
\section{Languages with hard constraints}\label{Disc:HardConst}

This study showed a variety of statistical tendencies of particle choice, word order, and intonation in Japanese.
Especially, in Chapter \ref{WordOrder} and \ref{Intonation},
I discussed the distinction between
elements that appear close to the predicate (in terms of word order) and are glued to the predicate (in terms of intonation) and
elements that appear separately from the predicate (in terms of both word order and intonation).
In this section,
I discuss other languages that have conventionalized the statistical tendency shown in this study.
As \citeA{bresnanetal01} state,
``soft constraints mirror hard constrains'';
namely, ``[t]he same categorical phenomena which are attributed to hard grammatical constraints in some languages continue to show up as statistical preferences in other languages,
motivating a grammatical model that can account for soft constraints'' (p.~29).
See also \cite{givon79,bybeehopper01}.

In \S \ref{Disc:HardConst:Integrated},
I discuss languages that
integrate some elements into the predicate.
In \S \ref{Disc:HardConst:Separated},
I focus on languages that
separate some elements from the predicate.

%%----------------------------------------------------
\subsection{Elements glued to the predicate}\label{Disc:HardConst:Integrated}

There are two kinds of elements proposed in this study
that are glued to the predicate:
strongly evoked elements that are postposed and
focus elements.

%%----------------------------------------------------
\subsubsection{Affixation of pronouns}

First, I discuss languages where strongly evoked elements, especially pronouns, are glued to the predicate.
As discussed in \S \ref{WOPostPreEles},
strongly evoked elements in spoken Japanese can appear immediately after the predicate, with a single intonation contour with the predicate.
This is a statistical tendency (i.e., soft constraint) rather than a categorical phenomenon (i.e., hard constraint),
showing that strongly evoked elements tend to be glued to the predicate.
I argue that in languages with hard constraints,
this corresponds to so-called ``grammatical agreement''.
In languages with grammatical agreement,
an affix, which is coreferential with the subject or the object,
typically attaches to the verb.
As \citeA[151]{givon76} states,
``[grammatical agreement and pronominalization] are fundamentally one and the same phenomenon, and [...] neither diachronically nor, most often, synchronically could one draw a demarcating line on any principled grounds.''
He argues that ``subject grammatical agreement'' arose from topic-shift constructions like \Next[a],
which are reanalyzed as ``subject-verb agreement'', as in \Next[b].
%
\ex.\label{Disc:HardConst:Integrated:Ex:TS}
 \a. Topic shift
 \bg.[] \EMi{The man}, \EM{he} came. \\
        (topic) (pronoun) (verb) \\
 \b. Neutral (reanalyzed)
 \bg.[] \EMi{The man} \EM{he}-came. \\
        (subject) (agreement)-(verb) \\
 \hfill{\cite[155]{givon76}}

Giv\'{o}n argues that
``[t]he morphological binding of the pronoun to the verb is an inevitable natural phenomenon, cliticization,
having to do with the unstressed status of pronouns, their decreased information load and the subsequent loss of resistance to phonological attrition'' (p.~155).
The following are examples from Swahili (Bantu).
In \Next[a],
the subject \ci{m-toto} `child (class 1)' has an agreement relationship with the verb prefix \ci{a} `he (class 1)'.
According to Giv\'{o}n,
the verb prefix \ci{a} originates from a pronoun.
Similarly, in \Next[b],
the subject \ci{ki-kopo} `cup (class 7)' agrees with \ci{ki} `it (class 7)'.
The examples are glossed based on \citeA{contini-morava94}.
%
\ex.
 \ag. \EM{m-toto} \EM{a}-li-kuja \\
      \ab{cl1}-child \ab{3}\ab{sbj}.\ab{cl1}-\ab{past}-come \\
      `The child came.'
% \bg. \EM{a}-li-kuja \\
%      \ab{3}\ab{sbj}.\ab{cl1}-\ab{past}-come \\
%      `He came (the child).'
% \bg. \EM{wa-toto} \EM{wa}-li-kuja \\
%      \ab{cl2}-child \ab{3}\ab{sbj}.\ab{cl2}-\ab{past}-come \\
%      `The children came.'
 \bg. \EM{ki}-kopo \EM{ki}-li-vunjika \\
      \ab{cl7}-cup \ab{3}\ab{sbj}.\ab{cl7}-\ab{past}-break \\
      `The cup broke.'
% \bg. \EM{ki}-li-vunjika \\
%      \ab{3}\ab{sbj}.\ab{cl7}-\ab{past}-break \\
%      `It broke (the cup).'
% \bg. \EM{vi}-kopo \EM{vi}-li-vunjika \\
%      \ab{cl8}-cup \ab{3}\ab{sbj}.\ab{cl8}-\ab{past}-break \\
%      `The cups broke.'
 \hfill{\cite[157]{givon76}}

Also, preposed objects are attested in Swahili,
and they have an agreement relationship with a verb affix in a way similar to subject agreement.
The object \ci{m-toto} `child (class 1)' agrees with the interfix \ci{kw} `him (class 1)', as in \Next[a], and
the object \ci{ki-kopo} `cup (class 7)' with \ci{ki} `it (class 7)', as in \Next[b].
%
\ex.
 \ag. \EM{m-toto}, ni-li-\EM{mw}-ona \\
      \ab{cl1}-child \ab{1}\ab{sg}-\ab{past}-\ab{3}\ab{obj}.\ab{cl1}-see \\
      `The child, I saw him.'
 \bg. \EM{ki-kopo}, ni-li-\EM{ki}-vunja \\
      \ab{cl7}-cup \ab{1}\ab{sg}-\ab{past}-\ab{3}\ab{obj}.\ab{cl7}-break \\
      `The cup, I broke it.'
      \hfill{(ibid.)}
      
\citeA{wals-101} states that
``[l]anguages in which pronominal subjects are expressed by pronominal affixes are widespread throughout the world.''
According to him,
in 437 out of 711 languages,
``pronominal subjects are expressed by affixes on verbs.''
Mian (Ok, Papua New Guinea) is one of those languages.
As shown in \Next, in Mian,
the subject is expressed by the suffix \ci{i}, and
the object are expressed by the prefix \ci{a}.
%
\exg.
 \EM{n\={e}} \EMi{naka=e} \EMi{a}-tem\^{e}'-b-\EM{i}=be \\
 \ab{1}\ab{sg} man=\ab{sg}.\ab{m} \ab{3}\ab{sg}.\ab{m}.\ab{obj}-see.\ab{impfv}-\ab{impfv}-\ab{1}\ab{sg}.\ab{sbj}=\ab{decl} \\
 `I am looking at the man.'
 \hfill{\cite[261]{fedden07}}


\citeA{givon76} argues that
the subject-agreement stems from topic-shift constructions like \ref{Disc:HardConst:Integrated:Ex:TS},
while the object-agreement originates from afterthought-topic constructions like \ref{Disc:HardConst:Integrated:Ex:AT},
i.e., post-predicate constructions,
at least in SVO languages.
%
\ex.\label{Disc:HardConst:Integrated:Ex:AT}
 \a. Topic shift
 \b.[] \EMi{The man}, I saw \EM{him}.
 \b. Afterthought
 \b.[] I saw \EM{him}, \EMi{the man}.
 \b. Neutral
 \b.[] I saw-\EM{him} \EMi{the man}.

\chd{Deaccented pronouns in Japanese can be interpreted as premature pronominal affixes.}

%\ex.
% \ag. y-a-bonye umu-nhu \\
%      \ab{3}\ab{sg}-\ab{past}-see \ab{cl1}-man \\
%      `He saw a man'
% \bg. y-a-\EM{mu}-bonye umu-nhu \\
%      \ab{3}\ab{sg}-
% \hfill{\cite[p.~159, sic]{givon76}}
%I suggest that post-predicate subjects could also evolve into
%subject suffix,
%since, in Japanese, subjects (A and S) can also be post-posed
%as shown in examples in \S \ref{WOPostPreEles}.
%However, according to Giv\'{o}n's prediction,
%``languages which use zero anaphoric pronouns[,] and in particular do not use anaphoric pronouns in topic-shift constructions, will not develop subject-verb or object-verb agreement'' ('p.~151);
%therefore, Japanese will not develop an agreement system in the future and
%my hipothesis cannot be tested.


%%----------------------------------------------------
\subsubsection{Noun incorporation}

While focus elements tend to be produced pre-predicatively in a coherent intonation contour with the predicate in Japanese,
I propose that, in languages with hard constraints,
focus elements are incorporated into the predicate.
In this section,
I point out some similarities between focus elements in the predicate focus environment and incorporated nouns.
Also, I discuss similarities between focus zero-coding and noun incorporation based on \citeA{mithun84}.
In noun incorporation,
a nominal and predicate form a unit;
nominals and the predicate are phonologically, morphologically, and syntactically {cohesive}.
According to \citeA{mithun84},
zero-coding is the first stage of noun incorporation.
%The following are characteristics of NI
%proposed in \citeA{mithun84,baker88,haspelmathsims10}.
%\ex.
%	\a. Nominals are (typically) {non-definite} (\S \ref{IndefN}). \hfill{\cite[p.\ 856]{mithun84}}
%	\b. various Nominal markers such as case markers, definite/indefinite markers, and number markers are dropped (\S \ref{MarkerDrop}). \hfill{(ibid.)}
%	\b. Nominals form {a unit with the predicates} (\S \ref{Unit}). \hfill{(ibid.)}
%	\b. Nominals and the predicate are phonologically, morphologically, and syntactically {cohesive} (\S \ref{Unit}). \hfill{\cite[p.\ 190ff.]{haspelmathsims10}}
%	\b. An {patient} S is more often incorporated
%		than an {agent} S (\S \ref{Hierarchy}). \hfill{\cite[p.\ 81ff.]{baker88}}
%	\b. Nominals lose its status as argument (\S \ref{NonArgument}). \hfill{\cite[p.\ 849]{mithun84}}

First,
as \citeA{mithun84} states,
typically incorporated nouns are {indefinite} and/or {non-specific},
which are features correlating with focus.
Definite and/or specific nouns, which are closer to topics,
are not incorporated into the verb.
Examples are shown below from Onondaga.
\citeA[11]{woodbury75a} states that
``[i]t is generally agreed that a noun which is incorporated makes a more general reference than one which is free of the verb stem.''
In \Next[a], the noun `tobacco',which is not incorporated into the verb,
refers to specific tobacco, and, as the translation shows,
it is interpreted as definite.
On the other hand, in \Next[b],
the incorporated noun `tobacco' refers to tobacco in general rather than a specific tobacco, as the translation shows.
%
\ex. Onondaga (Iroquoian)
 \ag. \tp{waP-ha-hnin\'{u}-P} \tp{neP} \tp{oy\'{E}Pkwa-P} \\
       \ab{tr}-\ab{3}\ab{sg}-buy-\ab{asp} the \EM{tobacco}-n.s. \\
       `He bought the tobacco.'
 \bg. \tp{waP-ha-yEPkwa-hn\'{\i}:nu-P} \\
      \ab{tr}-\ab{3}\ab{sg}-\EM{tobacco}-buy-\ab{asp} \\
      `He bought tobacco.'
      \hfill{\cite[10]{woodbury75a}}
%\ex. Ponapean (Oceanic)
% \ag. {i} {kanga-la} {\EM{wini}-o} \\
%	I eat-\ab{pfv} {medicine}-that \\
%	`I took all that medicine.'
% \bg. {i} {keng-\EM{winih}-la} \\
%	I eat-{medicine}-\ab{pfv} \\
%	`I completed my medicine-taking.' \hfill{\cite[214]{rehg81}}
%\ex. Kanjobal (Mayan)
%\ag. {\v{s}-{\O}-a-lo-t-oq} {\EM{in-pan}} \\
%	\ab{past}-\ab{3}\ab{abs}-\ab{2}\ab{erg}-eat-go-{\sc opt} \ab{1}\ab{erg}-{bread} \\
%	`You ate my bread.'
%\bg. {\v{s}-at-lo-w-i} \EM{pan} \\
%	\ab{past}-\ab{2}\ab{abs}-eat-{\sc aff}-{\sc aff} {bread} \\
%	`You ate bread.' \hfill{\cite{robertson80}}
%%% 要ページ数
%In Lahu,
%putting the accusative marker as in \Next[a] ``implies something like
%`drink the liquor in question; drink some particular liquor;
%drink liquor as opposed to something else' \cite[307]{matisoff81}.
%On the other hand,
%sentence without the accusative marker simply means
%`drink liquor' in general.
%%
%\ex. Lahu (Tibeto-Burman)
% \ag. \tp{j1} \tp{th\`{a}'} \tp{d\`{O}} \\
%      \EM{liquor} {\ab{acc}} drink \\
%      `to drink (the) liquor'
% \bg. \tp{j1} \tp{d\`{O}} \\
%      \EM{liquor} drink \\
%      `to drink liquor'
%      \hfill{\cite[307]{matisoff81}}

Similarly, in pseudo-noun incorporation in Niuean (Oceanic),
definite nouns cannot be incorporated into the verb.
Niuean is a VSO language;
canonically, the object appears after the subject.
On the other hand,
incorporated objects appear after the verb (before the subject),
from which one can see noun incorporation.
Unlike typical noun incorporation,
incorporated nouns can accompany modifiers,
as shown in \Next.
This is why \citeA{massam01} calls this pseudo-noun incorporation.
Note that the A argument \ci{mele} is coded as absolutive instead of ergative.
%
\ex. Niuean (Oceanic)
 \ag. ne inu \EM{kofe} \EM{kono} a mele \\
      \ab{past} drink coffee bitter \ab{abs} Mele \\
      `Mele drank bitter coffee.'
 \bg. ne holoholo \EM{kapiniu} \EM{kiva} fakaeneene a sione \\
      \ab{past} wash dish dirty carefully \ab{abs} Sione \\
      `Sione washed dirty dishes carefully.'
  \hfill{\cite[158]{massam01}}

Niuean does not allow nouns coded by case markers or number articles
to be incorporated
because they are interpreted as definite and non-specific.
%
\ex.
% \ag. *ne inu kofe a sione ne aute e au \\
%       \ab{past} drink coffee \ab{abs} Sione that made \ab{erg} \ab{1}\ab{sg} \\
%       `Sione drank coffee that I made.'
 \ag. *ne inu \EM{e} \EM{kofe} \EM{kona} a mele \\
      \ab{past} drink \ab{abs} coffee bitter \ab{abs} Mele \\
      `Mele drank the bitter coffee.'
 \bg. *kua holoholo \EM{tau} \EM{kapiniu} a mele \\
       \ab{pfv} wash \ab{pl} dishes \ab{abs} Mele \\
       `Mele washes the dishes.'
 \hfill{(op.cit.: 168)}

In Southern Tiwa,
all inanimate direct objects must be incorporated,
while animate direct objects are optionally incorporated \cite{allenetal84}.
As shown in the contrast between \Next[a] and \Next[b],
the inanimate object \ci{shut} `shirt' is incorporated,
otherwise it is ungrammatical.
%
\ex. Southern Tiwa (Tanoan)
 \ag. ti-\EM{shut}-pe-ban \\
      \ab{1}\ab{sg}.{\sc A}-shirt-make-\ab{past}\\
      `I made the/a shirt.'
 \bg. *\EM{shut} ti-pe-ban \\
      shirt \ab{1}\ab{sg}-make-\ab{past}\\
      \hfill{\cite[293]{allenetal84}}

On the other hand,
animate objects are only optionally incorporated,
they are grammatical irrespective of whether they are incorporated or not,
as shown in \Next[a-b].
%
\ex.
 \ag. ti-\EM{seuan}-m\~{u}-ban \\
      \ab{1}\ab{sg}.{\sc A}-man-see-\ab{past} \\
      `I saw the/a man.'
 \bg. seuanide ti-m\~{u}-ban \\
      man \ab{1}\ab{sg}.{\sc A}-see-\ab{past} \\
      `I saw the/a man.'
      \hfill{\cite[294-295]{allenetal84}}

Southern Tiwa is sensitive to animacy instead of definiteness.
However Southern Tiwa is like Onondaga and Niuean discussed above
in the sense that
Ps with features correlating with focus are incorporated,
while Ps with features correlating with topic can be not incorporated.


Second,
while patient nouns tend to be incorporated into the verb,
agent nouns are not incorporated \cite{mithun84,baker88}.
In Southern Tiwa, for example,
the patient Ss, `dipper' and `snow', are incorporated, as in \Next,
while the agent S, `dog', cannot be incorporated, as in \NNext.
\ex. Southern Tiwa (Tanoan)
 \ag. {i-\EM{k'uru}-k'euwe-m} \\
	{\sc B}-{dipper}-old-\ab{pres} \\
	`The dipper is old.'
 \bg. {we-\EM{fan}-lur-mi} \\
	{\sc C}.\ab{neg}-{snow}-fall-\ab{pres}.\ab{neg} \\
	`Snow isn't falling. (It is not snowing.)' \hfill{\cite[300]{allenetal84}}
	
\ex.
 \ag. {\EM{khwienide}} {{\O}-teurawe-we} \\
	{dog} {\sc A}-run-\ab{pres} \\
	`The dog is running.'
 \bg. *{{\O}-\EM{khwien}-teurawe-we} \\
	{\sc A}-{dog}-run-\ab{pres} \\
	`The dog is running.' \hfill{(op.cit.: 299)}

This is parallel with Onondaga, as shown by the contrast between \Next and \NNext.
Patient S is incorporated into the verb, as in \Next,
while agent S cannot be incorporated, as in \Next[b].
Glosses are based on \citeA[87-89]{baker88}.
%
\ex. Onondaga (Iroquoian)
 \ag. \tp{ka-\EM{hsahePt}-ah\'{\i}-hw-i} \\
	\ab{3}\ab{n}-\EM{bean}-spill-\ab{caus}-\ab{asp} \\
	`Beans got spilled.'
	\hfill{\cite[15]{woodbury75}}

\ex.
 \ag. \tp{h-at\'{e}-Ps\'{e}:-P} \tp{neP} \tp{o-\EM{ts\'{\i}Pkt}-aP} \\
	\ab{3}\ab{m}{\sc S}-\ab{refl}-drag-\ab{asp} the {\sc pre}-\EM{louse}-{\sc suf} \\
	`The louse crawls.'
 \bg. *\tp{h-ate-\EM{tsiPkt\'{\i}}-Pse:-P} \\
	\ab{3}\ab{m}{\sc S}-\ab{refl}-\EM{louse}-drag-\ab{asp} \\
	`The louse crawls.'
	\hfill{(ibid.)}

\citeA[875]{mithun84} argues that,
verb-internally, incorporated nouns bear a limited number of possible semantic relationships to their host verbs.
This applies no matter whether the language is basically of the ergative, accusative, or agent/patient type.
She proposes the following hierarchy of possible noun incorporations in different languages.
Agent S and A are put in parentheses because they are not attested in Mithun's data.
The hierarchy implies that
languages which incorporate patient S can also incorporate P,
but not necessarily vice versa.
%
\ex. P $>$ patient S ($>$ agent S $>$ A)
%	\ex.
%		\a. If a language incorporates nominals of only one semantic case, they will be \EM{Ps}.
%		\b. If a language incorporates only two types of arguments, they will be \EM{Ps and patient Ss}.
%		\b. Many languages additionally incorporate \EM{instruments and/or locations}.


I point out that the same hierarchy \Last explains the variety of zero-coding cross-linguistically.
According to \citeA{mithun84},
simple juxtaposition of a noun (without any markers) and a verb is the first stage of noun incorporation.
%First, there are languages which allow zero marking for Ps.
%Second, there are languages which allow zero marking for Ps and patient Ss.
%	\ex.\label{ZeroHierarchy}
%		\a. If a language zero-marks nominals of only one semantic case, they will be \EM{Ps}: Turkish, Hebrew, Romanian, and Sinhalese are examples of this. \label{OnlyPZero}
%		\b. If a language zero-marks only two types of arguments, they will be \EM{Ps and patient Ss}: Japanese and Lahu are examples of this. \label{PandS}
%		\b. If a language zero-marks only three types of arguments, they will be \EM{As and agent Ss}: Kansai Japanese and Korean are examples of this.
%		\b. Many languages additionally incorporate \EM{instruments and/or locations}.
There are many examples of languages without P-coding discussed in the literature
\cite[\ci{inter alia}]{comrie79,comrie83,croft03,aissen03,haspelmath08}.
In these languages,
Ps with features correlating with topic,
i.e.,  animate, human, and/or definite Ps,
are overtly coded,
while Ps with features correlating with focus
are zero-coded.
Some examples are discussed above as \ref{Disc:Ex:Mongolian1}-\ref{Disc:Ex:Mongolian2} in \S \ref{Disc:CompMotivations}.
Another example is from Russian,
which has a special marker for animate (or human) Ps,
but not for inanimate Ps.
As shown in the following examples,
\ci{nosorog} `rhinoceros' in \Next[a], an animate P,
is overtly coded by the direct object marker \ci{a},
whereas \ci{il} `slime', an inanimate P,
is zero-coded.
%
\ex.
 \ag. begemont ljubit nosorog\EM{-a} \\
      hippopotamus loves rhinoceros-\ab{do} \\
      `The/a hippopotamus loves the/a rhinoceros.'
 \bg. begemont ljubit il \\
      hippopotamus loves slime \\
      `The hippopotamus loves (the) slime.'

Examples for languages without P- and patient-S-codings are (Standard) Japanese and Lahu.
In (Standard) Japanese,
as discussed in \S \ref{Par:CasePar:Ga},
the agent S tends to be coded overtly, as in \Next[a],
while the patient S tends to be zero-coded, as in \Next[b-c]
\cite[93]{kageyama93}.

\ex.\label{Disc:Ex:Kodomo}
	\ag.\label{Disc:Ex:Kodomoa}a kodomo-\{ga/\EM{??\O}\} ason-deru \\
		oh child-\{\ab{nom}/\EM{\O}\} play-\ab{prog} \\
		`Look! A child is playing.' \hfill{(Agent S)}
%		\bg. a anna toko-ni kodomo-\{ga/\EM{\O}\} iru \\
%			oh that place-\ab{loc} child-\{\ab{nom}/\EM{\O}\} is \\
%			`Look! A child is in that kind of (dangerous) place!'
%		\bg. a kodomo-\{ga/\EM{??\O}\} taore-teru \\
%			oh child-\{\ab{nom}/\EM{\O}\} fall-\ab{prog} \\
%			`Look! A child has fallen (and he is lying).' \hfill{(patient S)}%animacy?
		\bg. a saihu-\{ga/\EM{\O}\} oti-teru \\
			oh wallet fall-\ab{prog} \\
			`Look! A wallet has fallen!' \hfill{(Patient S)}

\Next and \NNext are examples from Lahu.
As in \Next[a],
the definite P `the liquor' is coded with the accusative marker,
while the indefinite P `liquor' is not.
\ex. Lahu (Tibeto-Burman)
 \ag. \tp{j\`{1}} {\tp{th\`a'}} \tp{d\`O} \\
	\EM{liquor} \ab{acc} drink \\
	`to drink (the) liquor'
 \bg. \tp{j\`{1}} \tp{d\`O} \\
	\EM{liquor} drink \\
	`to drink liquor' \hfill{(P)}
 \b.[] \hfill{\cite[p.\ 307]{matisoff81}}

As in \Next,
the indefinite patient S is also zero-coded in Lahu (ibid.).
	\exg. \tp{m\^{u}-y\`e}\footnotemark{} \tp{l\`a} \\
			\EM{rain} comes \\
			`it is raining.' \hfill{(Patient S)}

 \footnotetext{
 The expression \ci{m\^{u}-y\`e} as a whole means `rain (noun)';
 which originates from \ci{m\^{u}} `sky' and \ci{y\`e} `water'
 \cite[60]{matisoff81}.
 }
%
There are also languages which zero-code P, patient S, and agent S.
In Kansai Japanese, for example,
agent Ss can be also zero-coded in addition to Ps and patient S.
% (p.c.\ with S.-H. Park).
%\ex. Korean
%	\ag. \tp{mun-{\O}} \tp{tata} \\
%		door-{\O} close \\
%		`Close the door!' \hfill{(P)}
%	\bg. \tp{pap-{\O}} \tp{matOpta} \\
%		food-{\O} not.tasty \\
%		`The food tasts bad!' \hfill{(Patient S)}
%	\bg. \tp{tSOki} \tp{kojaNji-\{ka/{\O}\}} \tp{kanta} \\
%		there cat-\{\ab{nom}/{\O}\} go \\
%		`There a cat goes!' \hfill{(Agent S)}
\ref{Disc:Ex:Kodomoa} without \ci{ga} is acceptable in Kansai Japanese \cite[see also][]{nakagawa13m}.%
	\footnote{
	Although the form of the sentence is identical,
	the pitch accent is drastically different and it is easy to distinguish Standard Japanese from Kansai Japanese.
	Grammaticality judgements are of mine.
	}


%Third,
%according to \citeA[p.\ 856]{mithun84},
%incorporated nouns lose their status as arguments.
%For example, in Yucatec,
%the verb takes the transitive suffix \ci{ik}
%when the direct P `tree' is not incorporated into the verb as in \Next[a],
%while the verb takes intransitive suffix (zero morpheme)
%when the direct P is incorporated as in \Next[b].
%%%% 要文献
%\ex. Yucatec (Mayan)
% \ag. {t-in-\v{c}'ak-{\O}-\EM{ik}} \EM{\v{c}e'} \\
%	  \ab{incomp}-\ab{1}\ab{sg}-chop-it-\ab{impfv}.{\ab{tr}} tree \\
%	`I chop a tree.'
% \bg. {t-in-\v{c}'ak-\EM{\v{c}e'}-\EM{\O}} \\
%	\ab{incomp}-\ab{1}\ab{sg}-chop-tree-\ab{impfv}.{\ab{intr}} \\
%	`I wood-chop.' \hfill{\cite{bricker78}}
%
%In Niuean pseudo-noun incorporation,
%the ``subject'' is coded by the absolutive marker \ci{a} as shown in \Next[b],
%while, in non-incorporated construcion,
%the ``subject'' is coded by the ergative marker \ci{e} as in \Next[a].
%%
%\ex. Niuean (Oceanic)
% \ag. takafaga t\={u}mau n\={\i} \EM{e} ia e tau \EM{ika} \\
%      hunt always \ab{emph} \ab{erg} \ab{3}\ab{sg} \ab{abs} \ab{pl} fish \\
%      `He is always fishing.'
% \bg. takafaga \EM{ika} t\={u}mau n\={\i} \EM{a} ia \\
%      hunt fish always \ab{emph} \ab{abs} \ab{3}\ab{sg} \\
%      `He is always fishing.'
%      \hfill{\cite[69]{seiter80}}
%
%There may be parallel phenomena observed in more analytic languages.
%In Tongan, for example,
%A, in this case \ci{sione}, is marked by the ergative marker when the P `fish' is coded by the definite marker as in \Next[a],
%while it is marked by the absolutive marker when the P is indefinite as in \Next[b].
%This indicates that \Next[b] is intransitive rather than transitive and that
%`fish' might have lost its status as argument.
%\ex. Tongan (Oceanic)
%	\ag. {na'e} {kai} \EM{'e} {sione} {'a} {e} \EM{ika} \\
%		\ab{past} eat \ab{erg} John \ab{abs} \ab{def} {fish} \\
%		`John ate the fish.'
%	\bg. {na'e} {kai} \EM{ika} \EM{'a} {sione} \\
%		\ab{past} eat {fish} \ab{abs} John  \\
%		`John ate fish.' \hfill{\cite[pp.\ 257-258]{hopperthompson80}}
%In Hungarian,
%for example,
%the object conjugation appears when the P is definite as in \Next[a],
%while it disappears when the P is indefinite as in \Next[b-c] no matter whether the P is specific or non-specific.
%Glosses are modified from \citeA[872]{mithun84}.
%%
%\ex.Hungarian
%	\ag. {a} {fi\'u} \EM{olvassa} \EM{az} {\'ujs\'agot} \\
%		\ab{def} boy {read}.\ab{obj} \ab{def} {paper} \\
%		`The boy is reading the newspaper.'
%	\bg. {a} {fi\'u} \EM{olvas} {egy} {\'ujs\'agot} \\
%		\ab{def} boy \EM{read} \EM{a} {paper} \\
%		`The boy is reading a [specific] newspaper.'
%	\bg. {a} {fi\'u} {\'ujs\'agot} \EM{olvas} \\
%		\ab{def} boy {paper} {read} \\
%		`The boy is reading a newspaper.'
%		\hfill{\cite[118]{beseetal70}}
%
%
%
%	\exg. kinoo \EM{eego-o} \EM{benkyoo-{\O}} si-ta \\
%		yesterday \EM{English-\ab{acc}} \EM{study-{\O}} do-\ab{past} \\
%		`Yesterday I studied English. (lit. I did study English.)' \hfill{Zero marking: Japanese}
%
%Mithun further argues that
%when a transitive verb incorporates its direct P, then an instrument, location, or possessor may assume the vacated P role.
%When an intransitive verb incorporates its S, another argument may be advanced to S status.
%In Yucatec, for example,
%the semantically locative `my cornfield' takes the locative marker
%when the direct P `tree' is not incorporated as in \Next[a],
%it does not take the locative marker
%when the direct P is incorporated
%and the verb's suffix is still transitive as in \Next[b].
%%%% 要文献
%\ex. Yucatec (Mayan)
% \ag. {t-in-\v{c}'ak-{\O}-\EM{k}} {\v{c}e'} \tp{i\v{c}il} {in-kool} \\
%	\ab{incomp}-I-chop-it-\ab{impfv}.{\ab{tr}} tree in my-cornfield \\
%	`I chop a tree in my cornfield.'
% \bg. {t-in-\v{c}'ak-\v{c}e'-\EM{t}-ik} {in-kool} \\
%	\ab{incomp}-I-chop-tree-{\ab{tr}}-\ab{impfv} my-cornfield \\
%	`I clear my cornfield.' \hfill{\cite{bricker78}}
%
%There might be zero-coding examples where non-arguments are coded by nominative or accusative markers.
%In Japanese,
%there is an example attested
%where an idiomatic expression `belly stands' (which means `to get angry')
%seems to function as a predicate as a whole and takes a new subject as in \Next.
%	\exg. are-wa musiro \EM{ore-tati-ga} \EM{hara-{\O}} tat-ta-yo-ne \\
%		that-\ab{top} rather \ab{1}\ab{sg}-\ab{pl}-\ab{nom} \EM{belly-{\O}} stand-\ab{past}-\ab{par}-\ab{par} \\
%		`In that event, WE got angry (rather than you).' \hfill{(\code{chiba0432: 111.64-113.37})}
%
%However,
%other analyses are possible.
%For example,
%\ci{ga} could be analyzed as focus marker not coding S or A.
%The putative nominative marker \ci{ga} does not always code S and A;
%it sometimes codes focalized non-S/A nominals
%as has been discussed in \ref{Par:CasePar:Ga:GaFoc}.
%The fact that in \Last the narrow focus `we' reading is the only possible reading
%could indicate that `we' is still non-argument but just focalized.


%%----------------------------------------------------
\subsection{Elements separated from the predicate}\label{Disc:HardConst:Separated}

As discussed in \S\S \ref{Int:IUISUnitCorp:Topic} and \ref{Int:IUISUnitExp},
topics which have not been established are produced intonationally separate from the predicate.
This section explores the possibility of the existence of
languages with hard constraints,
i.e., languages that do not allow unestablished topics to appear together with the predicate or main clause.

I did not find languages which match this exact condition.
However, one of the related phenomena is that, in some languages,
indefinite non-generic NPs cannot in general be the subject;
they can only be the subject of existential constructions \cite[173ff.]{givon76}.
I assume that, in these languages,
the connection between the subject (A and S) and topic is so strong
that non-topical subjects are not allowed.
%However,
%Giv\'on does not provide any examples and the details are not clear.%
% \footnote{
% Giv\'on refers to \citeA{hetzron71},
% but I did not succeed in obtaining this article.
% }
%I leave this as an open question for future studies.
\chd{
Canonical pre-verbal subjects in many Bantu languages are inherently topical and
subjects cannot be focus in situ (See \citeA{downinghyman16} and works cited therein for the summary of information structure in Bantu languages).
For example, in Northern Sotho,
it is possible for the subject to appear in the canonical pre-verbal position, as in \Next[a]
or in the post-verbal position, as in \Next[b].}
%
\ex.
 \ag. \EM{Mo-nna} o ngwala le-ngwalo \\
      \ab{cl1}-man \ab{cl1} write \ab{cl5}-letter \\
      `The man is writing a letter.'
 \bg. Go fihla \EM{mo-nna} \\
      \ab{cl17} arrive \ab{cl1}-man \\
      Lit. `There arrives a man.'
     \hfill{\cite[171]{zerbian06}}

\chd{
It is ungrammatical to put \ci{wh}-words in the canonical pre-verbal position, as shown in \Next.}
%
\ex.
 \ag. *\EM{Mang} \EM{o} nyaka ngaka? \\
       who \ab{cl1} look.for \ab{cl9}.doctor \\
       Intended: `Who is looking of the doctor?'
 \bg. *\EM{O} nyaka ngaka \EM{mang}? \\
       \ab{cl1} look.for \ab{cl9}.doctor who \\
       Intended: `Who is looking for the doctor?'
     \hfill{\cite[69]{zerbian06}}

\chd{In many Bantu languages,
it appears that an NP must be introduced in a special clause of non-canonical VS order and,
only after that, can the NP be mentioned in a normal clause of canonical SV(P) order to bring the narrative forward.}

\chd{Also in French, which is a SV(P) language,
VS order is used to focalize the subject and the predicate typically expresses existence, emergence, and motion \cite{togoohki86}.
Because the inverted subject is focus,
the scope of negation is the subject, as shown in \Next[a], and
it is unnatural to provide alternatives incompatible with the subject, as in \Next[b].}
%
\ex.
 \ag. Dans cet immeuble n'habitent pas \EM{des} \EM{ouvriers} \EM{fran\c{c}ais}, mais des ouvriers espagnols. \\
      in this building not.live \ab{neg} some workers French but some workers Spanish \\
      `In this building, French workers do not live, but Spanish workers.'
 \bg. ??Dans cet immeuble n'habitent pas \EM{des} \EM{ouvriers} \EM{fran\c{c}ais}, mais dans l'autre immeuble. \\
      in this building not.live \ab{neg} some workers French but in the.other building \\
      `In this building, French workers do not live, but in the other building.'\\
      \hfill{\cite[3, translated by NN]{togoohki86}}

\chd{It is infelicitous to put more new elements after the inverted subject.
For example, \Next[a], which is a typical subject inversion, is acceptable,
whereas \Next[b], which is \Next[a] followed by another phrase `by French and Japanese educators', is not acceptable.}
%
\ex.
 \ag. Dans ce d\'{e}bat ont \'{e}t\'{e} discut\'{e}s \EM{probl\`{e}mes} \EM{de} \EM{l'\'{e}ducation} \EM{morale}. \\
      in this debate have be discuss problems of the.education moral \\
      `In this debate, problems of moral education were discussed.'
 \bg. ??Dans ce d\'{e}bat ont \'{e}t\'{e} discut\'{e}s \EM{probl\`{e}mes} \EM{de} \EM{l'\'{e}ducation} \EM{morale} par des p\'{e}dagogues fran\c{c}ais et japonais. \\
      in this debate have be discuss problems of the.education moral by some educators French and Japanese \\
      `In this debate, problems of moral education were discussed by French and Japanese educators.'
      \hfill{(op.cit.: 4)}

\chd{Interestingly, however,
if a pause is inserted between the VS part (\ci{ont \'{e}t\'{e} discut\'{e}s \EM{probl\`{e}mes} \EM{de} \EM{l'\'{e}ducation} \EM{morale}} `problems of moral education were discussed') and the additional phrase (\ci{par des p\'{e}dagogues fran\c{c}ais et japonais} `by French and Japanese educators') in \Last[b],
the acceptability improves.
This suggests that a new NP is introduced in a special construction of VS order, and additional new information cannot be introduced within the same intonational phrase in French.}

%%----------------------------------------------------
\section{Summary}

This section outlined a summary of the study and
discussed languages that grammaticalize the tendencies proposed in this study.
Of course the discussion provided more possibilities than conclusion.
Further investigation is needed to analyze the exact associations between languages with hard constraints and those with soft constraints.
Also, it is intriguing to account for 
the factors that determine whether a language has
hard constraints or soft constraints.













