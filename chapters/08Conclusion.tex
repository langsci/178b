% !TEX root = ../main.tex
\chapter{Conclusion}\label{Conclusion}


%%----------------------------------------------------
%%----------------------------------------------------
\section{Summary}

This thesis attempted to answer partially to a larger question of
how Japanese speakers communicate with each other
through abduction of the mental state of other people.
It revealed that Japanese speakers employ a variety of cues to express the speaker's assumption about the hearer's mental state.

While a great amount of literature has discussed the distinction between \ci{wa} and \ci{ga},
the relationships among other kinds of particles have not been discussed.
Chapter \ref{Particles} in this thesis revealed the distinction between \ci{wa} and other topic particles such as \ci{toiuno-wa} and \ci{kedo/ga} preceded by copula,
as well as the distribution of case markers,
by drawing a semantic map of particles.
It also investigated the distribution of zero particles and their associations with information structure.

The previous literature investigated clause-initial, pre-predicate, and post-predicate constructions independently in different frameworks;
however, there was no unified account for word order in Japanese.
In Chapter \ref{WordOrder},
I described word order in spoken Japanese in a unified framework.

Chapter \ref{Intonation} investigated intonation.
While the previous literature mainly concentrates on contrastive focus,
this thesis discussed both topic and focus.
I investigated intonation as a unit of processing and
argued that information structure influences on the form of intonation units.

To the best of my knowledge,
particles, word order, and intonation in Japanese have been investigated separately in the literature;
there was no unified theory to account for the whole phenomena.
This thesis investigated the phenomena as a whole in a consistent way
by annotating the same information for all linguistic expressions and
by employing the same analytical framework.



%%----------------------------------------------------
%%----------------------------------------------------
\section{Theoretical and methodological implications}

This section discusses some theoretical and methodological implications
this thesis has.
First, I proposed that topic and focus are multidimensional
rather than homogeneous;
topic and focus are interpreted to be a bundle of features and
each feature is scalar rather than binary.
Different languages are sensitive to different features to different degrees.
Even within a language,
different linguistic expressions are sensitive to different features to various extents.
Moreover, it is often the case that
a single linguistic expression is sensitive to multiple features.
As outlined in Chapter \ref{Background},
different authors discuss different kinds of topic and focus,
which is a confusing situation.
I argue that linguistic research would be clearer if one asks
``what feature(s) is/are sensitive to what linguistic expression(s)?'',
instead of asking
``which feature best predicts the distribution of some linguistic expressions?''

Second,
I proposed methods of annotation and analysis
that are cross-linguistically applicable.
I did not annotate all the features proposed in \ref{ISFeatures} in \S \ref{FrameworkDefinition};
however, all the features can be defined independent of a language-specific categories and can be applied universally.
Some features such as specificity and definiteness are hard to annotate, and it is highly likely that different annotators have different intuitions about the expression in question.
I argue that this is not a problem.
In the real life,
some people might interpret some expression to be definite,
other people might interpret the same expression to be indefinite.
This is a source of linguistic variation, and there is no single right answer.
Ideally, a statistically sufficient amount of annotators annotate the same corpus, and all the annotations are used in analyses.

Third,
I point out the importance of qualitative analysis in addition to quantitative analysis.
In \S \ref{TopPar} in Chapter \ref{Particles}, for example,
I concluded that \ci{toiuno-wa} and \ci{wa} attaches to elements in different statuses in given-new taxonmy by examining each example,
even though the difference was not visible from raw numbers.
This is because my annotation is not fine-grained enough to capture the subtle difference between these markers.
Of course, it is necessary to run statistical tests in the future.
However, it is also important to examine each example to make sure that
the results do not contradict with observations.



%%----------------------------------------------------
%%----------------------------------------------------
\section{Remaining issues}

This thesis has many remaining issues to be investigated in the future.
I discuss two of these in this section.


%%----------------------------------------------------
\subsection{Predication or judgement types}

As discussed in Chapter \ref{Background},
traditional Japanese linguistics scholars have paid attention to
predication types or judgement types.
Predication or judgement types include the distinctions
\cite{matsushita28,yamada36,mio48,kuroda72,masuoka08,kageyama12}.
Although this thesis focused on the distinction among nominal types such as topic and focus,
the findings of this thesis can be integrated into theories of predication or judgement types.
This implies that information structure is not only related to properties of NPs;
rather, it is also associated with properties of predicates.
Especially, grammatical categories such as tense, aspect, modality, and evidentiality are highly likely related to types of information structure.
For example, as \citeA{masuoka12} points out,
the topic marker \ci{toiuno-wa} cannot be used in event predication (or stage-level predication);
it can only be used in property predication (or individual-level predication).%
 \footnote{
 See \S \ref{Back:GeneralChar:Toiunowa} in Chapter \ref{Background}
 for the distinction between property vs.~event predication.
 }
This is shown in the contrast between \Next[a] and \Next[b].
\Next[a], where \ci{toiuno-wa} is used in event predication with simple past tense,
is unacceptable.
In \Next[b],
on the other hand, where \ci{toiuno-wa} is used in property predication,
is acceptable.
%
\ex.
 \ag. *sachiko-\EM{toiuno-wa} uso-o tui-ta \\
      Sachiko-\ci{toiuno-wa} lie-\ci{o} spit-\ab{past} \\
      `Regarding Sachiko, she lied.'
     \hfill{\cite[p.~96]{masuoka12}}
 \bg. sachiko-\EM{toiuno-wa} uso-tuki-da \\
      Sachiko-\ci{toiuno-wa} lie--spitter-\ab{cop} \\
      `Regarding Sachiko, she is a lier.'
     \hfill{(Constructed)}

\citeA{masuoka12} concludes that
\ci{toiuno-wa} is used only for property predication.

Moreover, it is well known that
the interpretations of \ci{wa} and \ci{ga} changes
depending on predicate types \cite{kuroda72,kuno73}.
In property predication,
\ci{wa} is the default marker, and
\ci{ga} tend to be interpreted to be exhaustive listing.
As exemplified in \Next[a-b], both of which are copular sentences (i.e., property predication),
a sentence with \ci{wa} \Next[a] is considered to have a common topic-comment structure,
while a sentence with \ci{ga} \Next[b] is considered to focus only John,
namely, \Next[b] is interpreted as the answer to the question `who is a student?'
In Kuno's term,
\ci{ga} is interpreted to be exhaustive listing.
%
\ex.
 \ag. zyon-\EM{wa} gakusei-desu \\
      John-\ci{wa} student-\ab{cop} \\
      `John is a student.'
 \bg. zyon-\EM{ga} gakusei-desu \\
      John-\ci{ga} student-\ab{cop} \\
      `JOHN is a student. (it is John who is a student.)'
      \hfill{\cite[38]{kuno73}}

In event predication,
on the other hand,
\ci{ga} is the default marker and is interpreted to be neutral description,
while \ci{wa} tend to be interpreted as contrastive.
In \Next[a-b], which are event predication,
the NP followed by \ci{wa} in \Next[a] is interpreted to be contrastive,
while the whole sentence including the NP with \ci{ga} in \Next[b] is interpreted to have broad focus structure;
namely, in Kuno's term,
\ci{ga} is considered to be neutral description.
%
\ex.
 \ag. ame-\EM{wa} hut-te i-masu-ga... \\
      rain-\ci{wa} fall-and \ab{prog}-\ab{plt}-though \\
      `Though it does rain...'
 \bg. ame-\EM{ga} hut-te i-masu \\
      rain-\ci{ga} fall-and \ab{prog}-\ab{plt}-though \\
      `It is raining.'
      \hfill{(ibid.)}


As far as I notice,
there are few studies investigating the question of
why sentences of some information structure type are associated with
particular predication types.



%%----------------------------------------------------
\subsection{Genres}

Genres are also an important factor to influence the phenomena investigated in this thesis.
As pointed out in \ref{BackSubSubZero} (Chapter \ref{Background}),
for example,
the choice between zero vs.~overt particles is sensitive to
styles (casual vs.~formal).
However, it is not clear why the formal style requires overt particles than
the casual style.

Also, I have argued that
post-predicate constructions are more frequent in conversations than monologues.
Although I suggested a few possible answers why this is the case (\ref{WOPostPreEles} in Chapter \ref{WordOrder}),
there is still no clear answer.
Since there is a corpus of conversations annotated in the same way as this thesis \cite{nakagawaden12},
it is useful to compare these corpora.

In monologues like the ones employed in this thesis,
it is likely that predicate-focus structure appears more frequently than usual conversations;
\chd{the speaker usually talks about what s/he did or what happened to his/her in narratives, which fixes a topic (typically the speaker), and fixing a topic elicits predicate-focus structure.}
\chd{Moreover, because of the absence of hearers who ask \ci{wh}-questions and misunderstand what the speaker means,} the speaker less frequently has to answer \ci{wh}-questions or correct hearers, which typically elicit argument-focus structure.
It is important also for this reason to investigate other genres of spoken language.







