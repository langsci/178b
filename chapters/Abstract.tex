% !TEX root = ../main.tex
\addchap{Abstract}\label{Abstract}
\begin{refsection}

This thesis investigates the associations between information structure
and linguistic forms in spoken Japanese
mainly by analyzing spoken corpora.
It proposes multi-dimensional annotation and analysis procedures of spoken corpora and
explores the relationships between information structure
and particles, word order, and intonation.

Particles, word order, and intonation in spoken Japanese have been investigated separately in different frameworks and different subfields in the literature;
there was no unified theory to account for the whole phenomena.
This thesis investigated the phenomena as a whole in a consistent way
by annotating all target expressions in the same criteria and
by employing the same analytical framework.
Chapter \ref{Introduction} outlines the questions to be investigated and introduces the methodology of this thesis.
Chapter \ref{Background} reviews the literature of Japanese linguistics
as well as the literature on information structure in different languages.
Chapter \ref{Framework} proposes the analytical framework of the thesis.
Major findings are discussed in Chapter \ref{Particles}, \ref{WordOrder}, and \ref{Intonation}.

Chapter \ref{Particles} analyzes the distributions of topic and case particles.
It is made clear that so-called topic particles 
(\ci{wa}, zero particles, \ci{toiuno-wa}, and \ci{kedo/ga} preceded by copula) are mainly sensitive to given-new taxonomy,
whereas case particles (\ci{ga}, \ci{o}, and zero particles) are sensitive to both focushood and grammatical function.
While the distinction between \ci{wa} and \ci{ga} attract much attention in traditional Japanese linguistics,
the distribution of different kinds of topic and case particles, including zero particles,
are analyzed in this thesis.

Chapter \ref{WordOrder} studies word order:
i.e., clause-initial, pre-predicate, and post-predicate noun phrases.
Topical NPs appear either clause-initially or post-predicateively,
while focal NPs appear pre-predicatively.
Clause-initial and post-predicate NPs are different mainly in statuses in given-new taxonomy.
The previous literature investigated clause-initial, pre-predicate, and post-predicate constructions in different frameworks;
however, there was no unified account for word order in Japanese.
The thesis outlines word order in spoken Japanese in a unified framework.

Chapter \ref{Intonation} investigates intonation.
While the previous literature mainly concentrates on contrastive focus,
this thesis discusses in terms of both topic and focus.
It turns out that intonation corresponds to a unit of processing and
argues that information structure influences the form of intonation units.

Chapter \ref{Discussion} discusses theoretical implications of these findings.
Finally, Chapter \ref{Conclusion} summarizes the thesis
and points out some remaining issues and possible future studies.

\end{refsection}
