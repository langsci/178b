% !TEX root = ../main.tex
\chapter{Framework}\label{Framework}

%%% topic, focusに直接対応する語順、助詞、イントネーションなどはない
%%% 下位レベルの素性で決まる
%%% Persistenceの定義
%%% CSJ-RDV版の変数名が変更されている

%%----------------------------------------------------
%%----------------------------------------------------
\section{Introduction}\label{FrameworkIntro}

In this chapter I describe the framework I employ here.
First, in \S \ref{FrameworkSemanticMap},
I introduce the theory of conceptual space assumed throughout.
Then, I define the concepts of 'topic' and 'focus' I adopt, as well as describe the features
which have been proposed to be associated with information structure phenomena (\S \ref{FrameworkDefinition}).
Finally,
\S \ref{FrameworkTFIdent} explains the characteristics of the corpus to be investigated and how to annotate features correlating with topic and focus.

To investigate cognitive motivations of some linguistic category (e.g., topic and focus),
it is possible to use a variety of clues such as generalizations about typological tendencies, models of language processing, theories of language change and language contact, language acquisition process, and language production data, as well as traditional grammaticality and acceptability judgements of sentences.
%Although it is ideal to investigate all of these clues to construct a hypothesis about human cognition of language,
This study mainly employs language production data (a.k.a.\ corpora) and
the acceptability of sentences
because these two directly reflect the intuition and cognition of adult native speakers of Japanese.
Sometimes I also use production experiments to obtain enough data
under controlled contexts.
It is necessary to investigate other kinds of clues such as typological tendencies, language processing models, and language acquisition processes of many other languages
to reveal how cognition is reflected in human language in general.
I hope that this study contributes to this larger goal.

This study restricts itself to investigating only standard Japanese,
since large spoken corpora are available in this language.
%One of the reasons for this is that
There are few empirical studies on information structure in spoken Japanese,
while there are at least preliminary empirical studies in other languages, such as some European languages and languages in Africa \cite[e.g.,][]{cowles03,dipperetal04,dipperetal07,ritzetal08,skopeteasetal06,cookfildhauer11,chiarcosetal11}.
Another reason is that
a large spoken corpus of standard spoken Japanese is available.
% and few scholars have investigated information structure using this corpus.
The corpus is called \ci{the Corpus of Spontaneous Japanese} (CSJ) and 
is morphologically analysed and annotated with a variety of information such as accentual phrases, intonation, parts of speech, dependent structures in addition to basic transcriptions of speech \cite{maekawa03,maekawaetal04}.
I will describe characteristics of the corpus in \S \ref{FrameworkCorpus}.


%%----------------------------------------------------
%%----------------------------------------------------
\section{Conceptual space and semantic maps}\label{FrameworkSemanticMap}

Throughout this study,
I assume a theory of conceptual space \cite{croft01,haspelmath03}.
A conceptual space is a multi-dimensional model of concept sensitive to some linguistic function(s).
As \citeA[93]{croft01} states,
``conceptual space is a structured representation of functional structures and their relationships to each other. [...] Conceptual space is also multidimensional, that is, there are many different semantic, pragmatic, and discourse-functional dimensions that define any region of conceptual space''.
It is claimed to be universal.
An example of conceptual space is shown in Figure \ref{FrameworkCSF}.
This is a conceptual space of parts of speech.
The horizontal dimension given in capital letters indicates
``the constructions used for the propositional acts of reference, modification, and predication'' \cite[][p.\ 93]{croft01}.
The vertical dimension indicates the semantic classes of ``the words that fill the relevant roles in the propositional act constructions''  (op.cit.: 94).

Whereas
``the conceptual space is the underlying conceptual structure,
[...] a semantic map is a map of language-specific categories on the conceptual space'' (p.\ 94).
While conceptual space is supposed to be universal,
semantic maps are language-specific.
Figure \ref{FrameworkSMF} is an example of a semantic map of parts of speech specific to Japanese.
The dimensions are suppressed for the purpose of convenience.
The figure shows that
nouns such as \ci{hon} `book' accompany \ci{no} to modify another noun and \ci{da} for predication. %(58-59)
Adjectives such as \ci{yasu} `cheap' accompany \ci{i} for both modification and predication. %(60-61)
Some nominal adjectives between `book' and `cheap' such as \ci{heewa} `peace(ful)' and \ci{kenkoo} `health(y)' accompany both \ci{no} and \ci{na} for modification and \ci{da} for predication.
They are different from but similar to nouns such as `book'.
Some nominal adjectives such as \ci{atataka} `warm' and \ci{tiisa} `small' accompany both \ci{na} and \ci{i} for modification, and
`warm' allows both \ci{da} and \ci{i} to follow in predication.
This indicates that
they are similar to adjectives rather than nouns.
The nominal adjective \ci{kirei} `pretty' is in between;
it only allows \ci{na} for modification and \ci{da} for predication.

\begin{figure}
 \fittable{
 \begin{tabular}{lclclclc}
             & \textsc{Reference} &  & \textsc{Modification} &  & \textsc{Predication} & & \\
 \textsc{Objects}     & Object    &  & Object       &  & Object      &  & Identity \\
\hhline{~~-~-~-~}
             & reference &  & modifier     &  & predication &  & predication \\
             & \hfill $|$ \hfill &  &  \hfill $|$ \hfill &  &  \hfill $|$ \hfill &  &  \\
 \textsc{Properties}  & Property  &  & Property     &  & Property    &  & Location \\
\hhline{~~-~-~-~}
             & reference &  & modifier     &  & predication &  & predication \\
             & \hfill $|$ \hfill &  &  \hfill $|$ \hfill &  &  \hfill $|$ \hfill &  &  \\
 \textsc{Actions}     & Action    &  & Action       &  & Action      &  &  \\
\hhline{~~-~-~~~}
             & reference &  & modifier     &  & predication &  &  \\
 \end{tabular}
 }
 \caption{Conceptual space for parts of speech \cite[92]{croft01}}
 \label{FrameworkCSF}
\end{figure}

\begin{figure}
 \centering
 \begin{tabular}{clll}
                     &              & \textsc{Modification} & \textsc{Predication} \\
 \textsc{Objects}    & `book'       & \cellcolor[gray]{.8} \ci{no}         & \cellcolor[gray]{.95}  \ci{da} \\
  \vdots             & `peace(ful)' & \cellcolor[gray]{.75} \ci{no}/\ci{na} & \cellcolor[gray]{.95}  \ci{da} \\
  \vdots             & `health(y)'  & \cellcolor[gray]{.75} \ci{no}/\ci{na} & \cellcolor[gray]{.95}  \ci{da} \\
  \vdots             & `pretty'     & \cellcolor[gray]{.85} \ci{na}         & \cellcolor[gray]{.95}  \ci{da} \\
  \vdots             & `warm'       & \cellcolor[gray]{.8} \ci{na}/\ci{i}  & \cellcolor[gray]{.87} \ci{da}/\ci{i} \\
  \vdots             & `small'      & \cellcolor[gray]{.8} \ci{na}/\ci{i} & \cellcolor[gray]{.9} \ci{i} \\
 \textsc{Properties} & `cheap'      & \cellcolor[gray]{.9} \ci{i} & \cellcolor[gray]{.9} \ci{i} \\
 \end{tabular}
 \caption{The semantic map for the Japanese Nominal, Nominal Adjective, and Adjective constructions \cite[95]{croft01}}
 \label{FrameworkSMF}
\end{figure}

``The hypothesis of typological theory, including Radical Construction Grammar,
is that most grammatical domains will yield universals of the form-function mapping that can be represented as a coherent conceptual space'' (p.\ 96), which is explicitly stated in \Next.
%
\ex. \label{SemanticMapHyp}\tl{Semantic Map Connectivity Hypothesis}: any relevant language-specific and construction-specific category should map onto a \EM{connected region} in conceptual space. \hfill{(ibid.)}

Japanese parts of speech in Figure \ref{FrameworkSMF} support this hypothesis.
For example,
morphemes such as \ci{no} and \ci{na} map onto different but connected regions on the conceptual space.
If the adjective suffix \ci{i} could also attach to \ci{hon} `book', but not to \ci{kirei} `pretty', for example,
this would be a counter-example to the hypothesis.

There are also conceptual spaces for information structure,
and I aim here to describe semantic maps of information structure in Japanese.
In terms of the theory of conceptual space,
each feature that has been proposed to correlate with information structure (to be discussed in the next section) is considered to be a dimension of the conceptual space.
Hence, the question I am pursuing here can be restated as follows:
what dimensions Japanese is sensitive to, and
how linguistic forms (i.e., particles, word order, and intonation) in Japanese map onto the semantic map of information structure in Japanese.

In the following section,
I outline the definitions of topic and focus I adopt and
the features correlating with topic and focus
which are considered to be dimensions of conceptual space for information structure.

%%----------------------------------------------------
%%----------------------------------------------------
\section{Topic, focus, and correlating features}\label{FrameworkDefinition}

It has been pointed out that
there is a correlation between a topic and
a referent that is activated, definite, specific, animate, agent, and inferable,
and between a focus and
a referent that is inactivated, indefinite, non-specific, inanimate, and patient
\cite{givon76,keenan76,comrie79,comrie83}.
They form a prototype category;
e.g., topics are typically (i.e., frequently) but not always definite or animate, and
foci are typically but not always indefinite or inanimate.
%In addition to these features,
%I propose that
%the correlation between a topic and
%a referent that is inferable, specific, and entity,
%and between a focus and
%a referent that is non-inferable, non-specific, and proposition.
I propose that
the feature \fea{presupposed} is a necessary feature of topic,
while the feature \fea{asserted} is a necessary feature of focus.
On the other hand, other features correlate with topic and focus respectively but are not necessarily topics or foci themselves.
The features correlated with topic and focus are summarized in \Next.
\ex.\label{ISFeatures}
\begin{tabular}{lll}
	 & topic & focus \\
	a. & presupposed & asserted \\
	b. & evoked & brand-new \\
	c. & definite & indefinite \\
	d. & specific & non-specific \\
	e. & animate & inanimate \\
	f. & agent & patient \\
	g. & inferable & non-inferable \\
%	h. & entity & proposition \\
\end{tabular}

As will be shown in the following chapters,
topic and focus are heterogeneous
and have complex features proposed in \ref{ISFeatures}.

In this section,
I will define each term in \Last.

%%----------------------------------------------------
\subsection{Topic}\label{FrameworkTopic}

A linguistic form is considered to represent a topic
if it has the characteristics as in \ref{BackDefTopic} in \S \ref{BackSubsecDefTopic},
here repeated as \Next.
%
\ex.\label{FrameworkTopicDef}Topic is a discourse element that the speaker assumes or presupposes to be shared (known or taken for granted) and uncontroversial in a given sentence both by the speaker and the hearer.
%and that can be at issue at the time of utterance.
%	That is, topic is an element that is presupposed.

%While focus might also be activated,
%as discussed in \ref{FrameworkOthFeatures},
%being activated is not a definition feature of focus
%and if a linguistic form allows to represent a inactivated element as well as an activated element,
%it represents focus.
%
Since the proposition that ``the speaker assumes or presupposes to be shared both by the speaker and the hearer'' is too long and complicated,
this statement is sometimes shortened to ``shared by the speaker and the hearer'' to mean the same thing.
Remember that the statement is always the speaker's assumption
and hence avoids the paradox pointed out in \citeA{clarkmarshall81}.
The topic is by definition presupposed to be shared both by the speaker and the hearer.
By ``topic is shared'',
I mean that topics are either evoked, inferable, declining, or unused
in terms of the given-new taxonomy \ref{Back:Top:DefTop:GNTaxonomy} in \S \ref{BackSubsecDefTopic}.
By ``topic is presupposed'',
I mean that the speaker assumes that the hearer takes it for granted that the referent or the proposition being mentioned is known or accepted both by the speaker and the hearer.
See also the discussion in \S \ref{BackSubsecDefTopic}.

Also, the notion of \ci{uncontroversial} is important;
topics cannot be questioned or argued against in a normal manner.
For instance,
English noun phrases preceded by \ci{as for} or \ci{regarding} cannot be questioned or argued against.
Assuming that expressions like \ci{regarding} and \ci{as for} introduce topic expressions \cite{kuno72,kuno76,gundel74},
this supports the idea that topics cannot be questioned or argued against.
In \Next, for example,
\ci{John} preceded by \ci{as for} or \ci{regarding} cannot be felicitously argued against as shown in \Next[B2,B2$^{\prime}$],
whereas \ci{a teacher}, which is considered to be focus,
can be argued against as in \Next[B2$^{\prime\prime}$].
%
\ex. \label{BackExJohn}\a.[A1:] Do you remember the guys we met at the last night's party?\label{BackExJohnA}
     Their names are Karl and John, I guess.
     Karl is doing linguistics at the grad school of our university.
     I forgot what languages he speaks.
     \b.[] [\{As for/Regarding\} John]$_{TOP}$, [he]$_{TOP}$ [is a teacher]$_{FOC}$.
     \b.[B2:] ??No, \EM{Rob} is a teacher.
     \b.[B2$^{\prime}$:] ??No, \{as for/regarding\} \EM{Rob}, he is a teacher.
     \b.[B2$^{\prime\prime}$:] No, John is \EM{an engineer}.
%%% 要確認

In other words, topic expressions cannot be corrected by the next speaker in a normal manner.
I call this type of test the \ci{no}-test
 (see also the lie-test in \citeA[39]{erteschik-shir07}).

Careful readers might think that it is perfectly natural to produce an utterance like \Next which is very similar to \Last[B2],
speculating that the \ci{no}-test is a flawed test.
The capital letters in \Next indicates that
those words are stressed.
%
\ex. \a.[B2:]\label{BackExRob} No, {ROB} is a teacher, not JOHN.

However, this does not mean that the test is flawed.
Note that the participants of this conversation would not be satisfied only with \Last;
John's information needs to be provided.
Therefore, a ``complete'' conversation is something like \Next.
%
\ex. \a.[A1:] [\{As for/Regarding\} John]$_{TOP}$, [he]$_{TOP}$ [is a teacher]$_{FOC}$. \hfill{(=\ref{BackExJohnA})}
     \b.[B2:] No, {ROB} is a teacher, not JOHN. \hfill{(=\ref{BackExRob})}
     \b.[A3:] Then what is John?
     \b.[B4:] I guess he is an engineer.

This suggests that
once B says \ci{no}, s/he must provide an alternative to the focus (as long as s/he knows).
I am inclined to label \ci{ROB} in \Last[B2] as focus
and think that the existence of examples like \LLast[B] does not invalidate the \ci{no}-test.

It is also unnatural to overtly receive topics as news
because overt acceptance indicates that they could be controversial.
For instance, as shown in \Next[B2],
topics cannot be repeated as news by the next speaker who has heard the utterance \Next[A1],
whereas there is no problem to repeat the focus as news as in \Next[B2$^{\prime}$].
%
\ex.\label{aha} \a.[A1:] [\{As for/Regarding\} John]$_{TOP}$, [he]$_{TOP}$ [is a teacher]$_{FOC}$.
     \b.[B2:] ??Aha, \EM{John}.
     \b.[B2$^{\prime}$:] Aha, \EM{a teacher}.
%%% 要確認

I call this test the \ci{aha}-test.
The \ci{aha}-test is a natural consequence of the fact that
the truth value of a sentence is assessed with respect to topic \cite{strawson64}.

Let us see specific examples of topics.
For instance,
as will be shown in Chapter \ref{Particles},
preposed zero-coded elements (elements without any overt particles) correspond to topics in Japanese
because the referent that the preposed element refers to is presupposed to be shared between the speaker and the hearer as \ci{nezumi} `mouse' in \Next,
where {\O} indicates ``a zero particle''.
	\ex. \label{FrameworkExMouse}Context: Y and H are roommates,
		who are bothered by a mouse running in their room
		and eating their leftovers.
		The cat they keep finally caught the mouse while H was out.
		When H is back, Y wants to let H know this news.
		\ag.[Y:] \EM{nezumi-\O} neko-ga tukamae-ta-yo \\
			nezumi-{\O} cat-\ci{ga} catch-\ab{past}-\ab{fp} \\
			`The cat caught (the) mouse.'
	
The referent `mouse' is interpreted as shared between the speaker and the hearer;
when the mouse is not shared between the speaker and the hearer as in \Next,
the utterance is infelicitous as shown by the contrast between \Next[Y] and \Next[Y$^{\prime}$].
	\ex. \label{TopDef}Context: Y and his cat is relaxing in the living room.
		H comes into the room.
		\a.[H:] Anything fun today?
		\bg.[Y:] ??\EM{nezumi-\O} neko-ga tukamae-ta-yo \\
			mouse-{\O} cat-\ci{ga} catch-\ab{past}-\ab{fp} \\
			Intended: `The cat caught a mouse.' \hfill{(=\LLast[Y])}
		\bg.[Y$^{\prime}$:] neko-ga \EM{nezumi-\O} tukamae-ta-yo \\
			cat-\ci{ga} mouse-{\O} catch-\ab{past}-\ab{fp} \\
			`The cat caught a mouse.'

When the mouse is not shared between the speaker (Y) and the hearer (H),
the preposed \ci{nezumi} `mouse' is infelicitous as in \Last[Y],
while \ci{nezumi} in the pre-predicate position is felicitous as in \Last[Y$^{\prime}$].

%Moreover, the referent `mouse' is presupposed to be potentially at issue.
%If Y and H start to live separately and Y utters \LLast to H one year later for the first time they meet,
%it is infelicitous.%
% \footnote{
% In this case, however, the speaker may well assume that the hearer has forgotten about the mouse because it is too trivial.
% Hence the speaker can treat the mouse as non-shared (new).
% In this case ``at-issue-ness'' is not necessary feature of topics.
% I leave this question for further study.
% }
%

Some readers might think that preposed zero-coded elements do not necessarily correspond to topics;
Instead, readers might suspect that they correspond to foci
because \ci{nezumi} `mouse' in \LLast is somehow ``new'' to the discourse,
or, more precisely,
it is not activated before the time of utterance \LLast[Y].
However, as discussed below,
foci are not subject to a constraint such that their referent must be assumed to be shared by the speaker and the hearer.
Typically, foci are indefinite referents that are not shared as specified in \ref{ISFeatures}.
Since the preposed zero-coded elements in Japanese do not refer to indefinite referents, as shown in \Last,
I categorize them as topics.

%%----------------------------------------------------
\subsection{Focus}\label{FrameworkFocus}

A linguistic form is considered to represent focus if it has the characteristics given in \ref{BackFocDef} in \S \ref{BackSubsecDefFocus},
repeated here as \Next for convenience.
%
\ex. Focus is a discourse element that the speaker assumes to be news to the hearer and possibly controversial.
S/he wants the hearer to learn the relation of the presupposition to the focus by his/her utterance.
In other words, focus is an element that is asserted.
\label{FocDef}

A focused discourse element is news in the sense that
the hearer is assumed not to know the relationships between the element and the presupposition.
For example,
consider the following example \Next.
\ex. \a.[Q:] Who broke the window?
	\bg.[A:] hanako-ga wat-ta-n-da-yo \\
		Hanako-\ci{ga} break-\ab{past}-\ab{nmlz}-\ab{cop}-\ab{fp} \\
		`HANAKO broke (it).'
	\b.[] Presupposition: ``x broke the window.''
	\b.[] Assertion: ``x = Hanako''

In \Last[A],
\ci{hanako} is shared in the sense that
her existence and identity are known by the speaker and the hearer.
However,
\ci{hanako} is also news in relation to the presupposition
``x broke the window'' at the time of utterance \Last[Q].
The speaker of \Last[A] lets the hearer learn the proposition that is assumed to be news: ``x = Hanako.''
\ci{Hanako} is the focus because
this is the part where the assertion is different from the presupposition.

I also emphasize that the speaker thinks that the focus might be \EMt{controversial}.
This implies that another participant of the conversation can potentially argue against the focus statement.
%the focus is assumed to be possibly controversial by the speaker,
%while the topic is not.
Therefore, the focus can be felicitously negated by the next speaker,
whereas the topic cannot.
This is exemplified in \ref{BackExJohn}, repeated here as \Next.
%
\ex. \label{BackExJohn2}\a.[A:] Do you remember the guys we met at the last night's party?
     Their names are Karl and John, I guess.
     Karl is doing linguistics at the grad school of our university.
     I forgot what languages he speaks.
     \b.[] [\{As for/Regarding\} John]$_{TOP}$, [he]$_{TOP}$ [is a teacher]$_{FOC}$.
     \b.[B:] ??No, \EM{Rob} is a teacher.
     \b.[B$^{\prime}$:] ??No, \{as for/regarding\} \EM{Rob}, he is a teacher.
     \b.[B$^{\prime\prime}$:] No, John is \EM{an engineer}.
%%% 要確認

As shown in \Last,
(part of) the focus \ci{a teacher} can be negated felicitously,
whereas the topic \ci{John} cannot be negated felicitously.
The concept of controversialness is more hearer-oriented and interactional than the previous notions such as assertions, unpredictability, and unrecoverablity.
See also the discussion in \S \ref{BackSecFocus}.
%The definition (\ref{FocDef}) is essentially the same as the definition of focus \Next proposed in \citeA{lambrecht94}.
%\ex. The semantic component of a pragmatically structured proposition
%	whereby the assertion differs from the presupposition.
%	\hfill{\cite[][p.\ 213]{lambrecht94}}

%%----------------------------------------------------
\subsection{Information structure in a sentence}\label{FrameworkIS}

%Predicate-focus structure
%Argument-focus structure
%Sentence-focus structure

Here I discuss types of information structure.
Following \citeA{lambrecht94},
I distinguish three types of information structures within a sentence:
\EM{predicate-focus structure} (topic-comment structure),
\EM{sentence-focus structure}, and
\EM{argument-focus structure}.
%In addition to these three types,
%I further divide predicate-focus structure into two types
%depending on the status of topics:
%\EM{the discontinuous topic} and \EM{the continuous topic}.

In \EM{the predicate-focus structure} or the topic-comment structure,
the predicate is the focus, as the name suggests.
The predicate may include the complement of the predicate.
This is exemplified in \Next[A] for English,
where the capital letters represent prominence in English.
\ex. Predicate-focus structure
	\a.[Q:] (What did the children do next?)
	\b.[A:] [The children]$_{T}$ [went to SCHOOL]$_{F}$.
	\hfill{\cite[][p.\ 121]{lambrecht94}}

\Next[A] is an example of predicate-focus structure in Japanese.
%
\ex.
 \a.[Q:] What is Hanako doing?
 \bg.[A:] [\EM{Hanako}-wa]$_{T}$ [syoosetu-o yon-deru]$_{F}$-yo \\
		Hanako-\ci{wa} novel-\ci{o} read-\ab{prog}-\ab{fp} \\
		`Hanako is reading a novel.'

%In this paper, I divide topics in predicate-focus structures into two types:
%discontinuous topic and continuous topic.
%They are exemplified in \Next[b] and \NNext[b] for English \cite[][p.\ 153, interpretation of information structure added]{givon76}.
%\ex. Predicate-focus structure with discontinuous topic
%	\a. Once there was \EM{a wizard}.
%	He was very wise, rich, and was married to a beautiful witch.
%	They had two sons.
%	The first was tall and brooding, he spent his days in the forest hunting snails, and his mother was afraid of him.
%	The second was short and vivacious, a bit crazy but always game.
%	\b. Now [\EM{the wizard}]$_{T}$, he [lived in Africa]$_{F}$.%
%		\footnote{
%		The status of \ci{he} in this example is irrelevant for the current discussion and hence is ignored.
%		}
%
%\ex. Predicate-focus structure with continuous topic
%	\a. Once there was \EM{a wizard}.
%	\b. [\EM{He}]$_{T}$ [lived in Africa]$_{F}$.
%
%\EM{The wizard} and \EM{he} in \LLast[b] and \Last[b] refers to \EM{a wizard} in \Next[a] and \NNext[a].
%However,
%they are intervened by other discourse referents in \Next,
%while they are not in \NNext.
%I will call the former discontinuous topic
%and the latter continuous topic.
%Examples in Japanese are shown in \Next[b] for discontinuous topic and \NNext[b] for continuous topic.
%\ex. Predicate-focus structure with discontinuous topic
%	\a. Taro and Hanako is in the room. Taro is sitting on the couch and playing a video game.
%	\bg. [\EM{hanako}-wa]$_{T}$ [hon(-o) yon-deru]$_{F}$-yo \\
%		Hanako-\ci{wa} book-\ci{o} read-\ab{prog}-\ab{fp} \\
%		`As for Hanako, she is reading a book.'
%
%\ex. Predicate-focus structure with continuous topic
%	\a. Hanako is in the room, and
%	\bg. [hon(-o) yon-deru]$_{F}$-yo \\
%		book-\ci{o} read-\ab{prog}-\ab{fp} \\
%		`She is reading a book.'


In \EM{the sentence-focus structure},
the whole sentence is focused.
This is exemplified in \Next[A] for English,
where, again, the capital letters indicate stress.
\ex. Sentence-focus structure
	\a.[Q:] What happened?
	\b.[A:] [The CHILDREN went to SCHOOL]$_{F}$!
	\hfill{\cite[][p.\ 121]{lambrecht94}}

A Japanese example of sentence-focus structure is shown in \Next[A].
\ex. Sentence-focus structure
	\a.[Q:] What happened? 
	\bg.[A:] [hanako-ga syoosetu(-o) yon-deru]$_{F}$-yo \\
		Hanako-\ci{ga} novel-\ci{o} read-\ab{prog}-\ab{fp} \\
		`Hanako is reading a novel!'

In sentence-focus structure,
there is no explicit topic and all the arguments (e.g., \ci{the children} and \ci{school} in \Last[A]) are part of the focus.
However, if one assumes stage topics \cite{erteschik-shir97,erteschik-shir07},
the distinction between the predicate-focus and the sentence-focus structures may not be clear.
%especially in languages with zero pronouns.
%In Japanese, for example,
%elements such as \ci{kyoo} `today' are coded by the so-called topic marker \ci{wa} and appear to function in a way similar to topics as in \Next[a].
In \Next[a], for example,
\ci{kyoo} `today' might function as a topic in the sense that
the truth value of the sentence is evaluated with respect to the specific time `today' (although, in this study, I do not examine stage topics in detail).
\ex. 
	\ag. [\EM{kyoo}-wa]$_{T?}$ [hanako-ga syoosetu(-o) yon-deru]$_{F}$-yo \\
		today-\ci{wa} Hanako-\ci{ga} novel(-\ci{o}) read-\ab{prog}-\ab{fp} \\
		`Today Hanako is reading a novel.'
		\bg. [\EM{Hanako}-wa]$_{T}$ [syoosetu-o yon-deru]$_{F}$-yo \\
		Hanako-\ci{wa} novel-\ci{o} read-\ab{prog}-\ab{fp} \\
		`Hanako is reading a novel.'

Note that, in terms of information structure,
\Last[a] is similar to \Last[b],
which has predicate-focus structure.
The predicate-focus and sentence-focus structures are similar
in that the predicate is in the domain of focus.
For this reason, I sometimes put the predicate-focus and sentence-focus structures into the same category
and refer to them as \EM{broad focus structures}.

In \EM{the argument-focus structure},
elements other than predicates are focused.
This is exemplified in \Next[A] for English
and \NNext[A] for Japanese.
This structure is sometimes referred to as the \EM{narrow focus structure}
as opposed to broad focus structure
because the domain of focus is limited to arguments or other elements except for predicates.
\ex. Argument-focus structure
	\a.[Q:] Who went to school?
	\b.[A:] [The CHILDREN]$_{F}$ [went to school]$_{T}$. \hfill{\cite[][p.\ 121]{lambrecht94}}

\ex. Argument-focus structure
	\a.[Q:] Who is reading a book?
	\bg.[A:] [hanako-ga]$_{F}$ [syoosetu(-o) yon-deru]$_{T}$-yo \\
		Hanako-\ci{ga} book(-\ci{o}) read-\ab{prog}-\ab{fp} \\
		`Hanako is reading a book.'

%Note that sentences where only the predicate is focused are of broad-focus structure.

%Definition of information, element, entity, etc.
I distinguish between two types of components constituting an information structure:
discourse element and discourse referent,
each of which is defined as in \Next:
\ex.
	\a. \EM{(Discourse) element}: {A unit of linguistic form (including zero pronoun) that is uttered by the participants in discourse.}
	\b. \EM{(Discourse) referent}: {An entity or proposition that a discourse element refers to (if a referent is a proposition, it is also called \EM{proposition}).}



%%----------------------------------------------------
\subsection{Other features correlating with topic/focus}\label{FrameworkOthFeatures}

This section discusses the definition of features which have been
proposed to correlate with topic and focus.
Although I do not necessarily annotate all the features in my corpus,
I discuss all of them
since, in some place or other, they are relevant to my proposals.

%%----------------------------------------------------
%\subsubsection{Presupposition vs. assertion}


%%----------------------------------------------------
\subsubsection{Activation cost}\label{FrameworkActivation}
The activation cost of a referent is the assumed cost for the hearer to activate the referent in question.
An active referent is a referent
that the speaker assumes to be in the attention of the hearer (and hence the activation cost is low),
while an inactive referent is a referent
that the speaker does not assume to be in the attention of the hearer (and hence the activation cost is high)
\cite[see also][inter alia]{chafe94}.%
	\footnote{
	I am using the term \ci{attention} rather than \ci{consciousness}
	because I believe the speaker's ability to evaluate the hearer's state of mind is eventually related to joint attention \cite{tomasello99}.
	}
Typically,
referents are assumed to be brought to the hearer's attention
by mentioning them or putting them in the hearer's area of visual perception.

A topic referent is often, but not always, activated in the hearer's mind.
In \ref{FrameworkExMouse},
the referent `mouse' is not necessarily considered to be active in H's mind.
Although the mouse kept bothering Y and H sometimes when they were in their room,
it is not appropriate for the speaker to assume that the mouse is in H's attention anymore in school when the speaker happened to talk to H.

According to \citeA{dryer96},
focus is an element that is not activated.
While this generalization well captures the view that the focus is the stressed linguistic element,
I will not employ this definition of focus in this study
because if \ci{nezumi} `mouse' in \ref{FrameworkExMouse} is focus,
one has to come up with an explanation for why it is assumed to be shared between the speaker and the hearer,
which is typically not the case with focus.
According to my account, on the other hand,
\ci{nezumi} `mouse' in \ref{FrameworkExMouse} is topic because the the characteristics are in accordance with topic correlation features in \ref{ISFeatures}
and a special account for why \ci{nezumi} `mouse' is shared is not necessary.
For detailed discussion of the relationships between focus and stress,
see \citeA[Chapter 5]{lambrecht94}.

A focus referent, on the other hand, is typically assumed not to be active in the hearer's mind.
As \citeA{lambrecht94} has pointed out,
the most frequent focus structure is predicate-focus structure as in \Next[A,B],
where elements included in the predicate focus are typically not active in the hearer's mind.
\ex.\label{tomodati} \a.[Q:] What did you guys do today?
	\bg.[A:] [watasi-wa]$_{T}$ [tomodati-to resutoran-de supagetii tabe-ta]$_{F}$-yo \\
			\ab{1}.\ab{sg}-\ci{wa} friend-with restaurant-\ab{loc} spaghetti eat-\ab{past}-\ab{fp} \\
			`I ate spaghetti with (a) friend in (a) restaurant.'
	\bg.[B:] [boku-wa]$_{T}$ [uti-de hon yon-de-ta]$_{F}$-yo \\
			\ab{1}.\ab{sg}-\ci{wa} home-\ab{loc} book read-\ab{prog}-\ab{past}-\ab{fp} \\
			`I was reading (a) book at home.'

In \Last,
it is reasonable to assume that
Q did not have `friend', `restaurant', `spaghetti', `home', and `book' in his/her attention at the time of utterance \Last[Q].
%I also use the term \EM{new} interchangeablly with ``inactivated.''

There is another type of activation status: \fea{semi-active}.
I use the term \ci{declining} specifically for the referent that has been active but starts to decline because other referents are also activated.
Declining elements are in semi-active state.

%%----------------------------------------------------
\subsubsection{Definiteness}\label{Fr:Definition:TFFeathers:Definite}

A definite referent is a referent
that is unique in the domain of discourse,
while an indefinite referent is a referent
that is not unique in the domain of discourse.

The claim that ``topic is a discourse element that the speaker assumes or presupposes to be shared (known or taken for granted) and uncontroversial in a given sentence both by the speaker and the hearer'' in \ref{FrameworkTopicDef} 
might lead to the interpretation that the topic is definite.
As has been pointed out in the literature \cite{givon76,keenan76,comrie79,comrie83},
topics tend to be definite.
However, this is not a necessary nor sufficient feature of topics.
Let us discuss the following example \Next.%
	\footnote{
	I am grateful to Yoshihiko Asao
	for pointing out this type of example.
	}
\ex.\label{Fr:Definition:TFFeathers:Definite:Ex:Mango1}Context:
	Y told H that he had never seen and eaten mangoes.
	H told Y that they are delicious.
	Several days later, Y finally ate a mango.
	\ag.[Y:] \EM{mangoo} konoaida miyako-zima-de tabe-ta-yo \\
			mango the.other.day Miyako-island-\ab{loc} eat-\ab{past}-\ab{fp} \\
			`(I) ate (a) mango (we talked about) in Miyako island the other day.'

In \Last `mango' is indefinite because the mango Y ate is not unique in the domain of discourse; H cannot uniquely identify which mango Y ate.%
 \footnote{
 \chd{Yuji Togo and one of the reviewers (Morimoto) cast doubt on my claim
      that \ci{mangoo} in \ref{Fr:Definition:TFFeathers:Definite:Ex:Mango1} is indefinite;
      Rather, they suggest that it could be generic.
      I am reluctant to accept this view because
      this \ci{mangoo} seems to refer to a specific (non-generic) mango
      that Y ate, as indicated by the past tense of the predicate \ci{tabe-ta} `eat-\ab{past}'.
%      On the other hand, I share the intuition that this preposed \ci{mangoo}
%      is not just a specific mango.
      }
 }
However, the element \ci{mangoo} `mango' is preposed because it has been discussed and hence is assumed to be shared between the speaker and the hearer.
This makes it possible for \ci{mangoo} to appear clause-initially as will be discussed in Chapter \ref{WordOrder}.
%This is exemplified more clearly
%in the contrast between \Next[Y2] and \Next[Y2$^{\prime}$].
%\ex.\label{Fr:Definition:TFFeathers:Definite:Ex:Mango2}Context:
%	Y and H have not met for a few months.
%	\a.[H1:] What did you do these days?
%	\bg.[Y2:] ??\EM{mangoo} konoaida miyako-zima-de tabe-ta-yo \\
%			mango the.other.day Miyako-island-\ab{loc} eat-\ab{past}-\ab{fp} \\
%		\hfill(=\LLast[Y])
%	\bg.[Y2$^{\prime}$:] konoaida miyako-zima-de \EM{mangoo} tabe-ta-yo \\
%			the.other.day Miyako-island-\ab{loc} mango eat-\ab{past}-\ab{fp} \\
%			`(I) ate (a) mango in Miyako island the other day.'
%
%In \Last,
%Y and H have not discussed mango before and hence preposing \ci{mangoo} `mango' as in \Last[Y2] is infelicitous,
%while that in the pre-predicate position as in \Last[Y2$^{\prime}$] is felicitous.
%Therefore,
%a topic is an element that is shared by the speaker and the hearer.
%It is frequently definite, but not necessarily.
%I categorize this kind of shared indefinites into \EM{unused} borrowing the term from \citeA{prince81}.
I include this type of example in the category of unused,
extending the term ``unused'' in \citeA{prince81}.

However,
some indefinite referents are more difficult to interpret as topics than others.
For example, expressions such as \ci{dareka} `somebody' and \ci{oozee-no hito} `many people' are poor candidates for topic than others
judging from the fact that they cannot be followed by \ci{wa}, but can be followed by \ci{ga} in Japanese as shown in \Next \cite[][p.\ 37 ff.]{kuno73}.
As will be shown in Chapter \ref{Particles},
\ci{wa} codes the element whose referent is assumed to be active in the hearer's mind;
\ci{wa} codes active topics.
On the other hand, as will also be shown in Chapter \ref{Particles},
\ci{ga} codes focus elements.
\ex. \a. \EM{dareka-\{??wa/ga\}} byooki-desu \\
		somebody-\ci{wa/ga} sick-\ab{cop}.\ab{plt} \\
		`Speaking of somebody, he is sick.'
	\b. \EM{oozee-no} \EM{hito-\{??wa/ga\}} paatii-ni ki-masi-ta \\
		many-\ab{gen} person-\ci{wa/ga} party-to come-\ab{plt}-\ab{past} \\
		`Speaking of many people, they came to the party.'



A focus referent, on the other hand,
tends to be indefinite rather than definite \cite{givon76,keenan76,comrie79,comrie83,dubois87}.
As has been mentioned above,
% in relation to example (\ref{tomodati}) in the previous section,
the most frequent focus structure is predicate-focus structure exemplified in \ref{tomodati} and
it is reasonable to assume that Q in \ref{tomodati} cannot identify the referents included in the predicate focus such as `friend', `restaurant', `spaghetti', and `book'.

%%%ほんと? to be modified.
%However, there is a tendency that argument focus is definite.
%For example, \Next in infelicitous.
%\ex. \a.[X:] Alice broke the window.
%	\b.[Y:] No. A MAN broke it.
%
%\ex. \a.[X:] A woman broke the window.
%	\b.[Y:] No. A MAN broke it.

It is natural for topic referents to be frequently realized by definite noun phrases.
The participants typically talk about the person or the thing whose identity is known by them.
Or sometimes they talk about people or something in more general terms.
This is an exceptional case known as a generic referent and requires a special account.
On the other hand, it is natural for focus referents to be frequently realized by indefinite noun phrases
because, intuitively, an element that is not known by the hearer in relation to a presupposition is typically not shared between the speaker and the hearer.
%They are independent persons and they encounter different people and things they do not share every day.


%%----------------------------------------------------
\subsubsection{Specificity}

A specific referent is fixed, namely, the speaker has one particular referent in his/her mind;
while a non-specific referent is not fixed,
i.e., the speaker does not have one particular referent in mind \cite{karttunen69diss,enc91,abbott94}.
Turkish unambiguously codes specific and non-specific objects:
if the NP is coded by the accusative case marker \ci{-(y)i} (or \ci{-(y)u}),
it is interpreted as specific as in \Next[a],
while, if the NP is not overtly coded,
it is interpreted as non-specific as in \Next[b].
\ex. \ag. Ali bir piyano-yu kiralamak istiyor \\
	Ali one piano-\ab{acc} to.rent wants \\
	`Ali wants to rent a certain piano.'
	\bg. Ali bir piyano kiralamak istiyor \\
	Ali one piano to.rent wants \\
	`Ali wants to rent a (non-specific) piano.'
	\hfill{\cite[][p.\ 4-5]{enc91}}

Specific referents like `piano' in \Last[a] are fixed
in the sense that
the speaker wants to rent a particular piano in his/her mind.
Non-specific referents like `piano' in \Last[b] are not fixed
in the sense that
the speaker does not care which piano s/he could rent;
any piano works in \Last[b].

Topics are frequently but not always specific.
Consider the following example \Next,
which is slightly modified from \ref{Fr:Definition:TFFeathers:Definite:Ex:Mango1}.
%
\ex.\label{Fr:Definition:TFFeathers:Specificity:Mango}Context:
	Y told H that he had never seen and eaten mangoes.
	H told Y that they are delicious.
	Several days later, Y finally got a chance to eat a mango.
	\ag.[Y:] \EM{mangoo} raisyuu miyako-zima-de taberu-yo \\
			mango next.week Miyako-island-\ab{loc} eat-\ab{fp} \\
			`(I will) eat (a) mango (we talked about) in Miyako island next week.'

In this case, \ci{mangoo} is non-specific because speaker Y does not know which mango he will eat.
However, it is the topic at the same time for the same reason discussed in association with \ref{Fr:Definition:TFFeathers:Definite:Ex:Mango1}.

There is a concept that is related to but distinct from non-specificity: genericity.
Generic referents do not represent an individual entity,
but do represent a concept or a category.
On the other hand, non-specific referents still represent an individual entity.
According to \citeA{kuno72},
generic referents are always available to be topic.
In \Next,
the element \ci{kuzira} corresponds to a generic referent as the topic.
%
\exg. \EM{kuzira}-wa honyuudoobutu-desu \\
		whale-\ci{wa} mammal-\ab{cop}.\ab{plt} \\
		`A whale is a mammal.' \hfill{\cite[][p.\ 270]{kuno72}}

When participants talk about generic referents,
the referent that is presupposed to be shared is the concept itself.
Therefore, generic referents are always shared
(unless the hearer has never heard the expression in question).
As will be shown in Chapter \ref{Particles}, however,
\ci{wa} codes the element whose referent is assumed to be active or semi-active inferable in the hearer's mind and
not all generic elements can be coded by \ci{wa}.


Foci, on the other hand, can be either specific or non-specific,
but tend to be non-specific.
In \Next[A],
the speaker may or may not have a particular book in his/her mind.
%
\ex. \a.[Q:] What are you going to do tomorrow?
	\b.[A:] [I]$_{T}$'m going to [read \EM{a book} tomorrow]$_{F}$.
%\ex.	 \a.[Q:] What did you do yesterday?
%	\b.[A:] [I]$_{T}$ [read \EM{a book} yesterday]$_{F}$.

In the example above,
the specificity of the book in question is not important.
Instead, the whole event of reading a book is more relevant to the question.
%In \Last[A], however,
%one can pragmatically infer from the past tense that
%there must be one specific book that the speaker read.



%%----------------------------------------------------
\subsubsection{Animacy}

An animate referent is a living entity such as human beings, cats, and dogs,
while an inanimate referent is a non-living entity such as computers, books, and love.
Snakes, bugs, plants, and flowers are somewhere in between.

Topic tends to be animate,
while focus tends to be inanimate \cite{givon76,keenan76,comrie79,comrie83,dubois87}.
Although this study does not discuss animacy very much,
it is relevant to some aspects of the distinction between
zero vs.\ overt particles,
as briefly mentioned in Chapter \ref{Particles}.


%%----------------------------------------------------
\subsubsection{Agentivity}

I employ the prototypes of the agent and the patient
discussed in \citeA[][inter alia]{dowty91}.
An agent is a referent
that typically has volition,
has sentience,
causes an event or change of state in another participant, or
moves.
On the other hand,
a patient is a referent
that typically undergoes a change of state,
corresponds to an incremental theme,
is causally affected by another participant, or
stationary relative to movement of another participant.

Agentivity or subjecthood is often discussed in association with topic \cite[][inter alia]{li76}.
%Some scholars even claim that agentivity is the definition of topic.
%%% Reference???
However, it is inaccurate to assume that a topic is limited to an agent
or that an agent is always the topic.
It is important to keep in mind that
topic correlates with agent or subject
but being an agent or subject itself is neither a necessary nor sufficient condition to be topic.
Focus, on the other hand, correlates with patients.
In the same way as topic, however,
it is inaccurate to assume that all foci are patients.
The relationships between topic/focus and agentivity are discussed in Chapter \ref{Particles},
in association with the distinction between zero vs.\ overt particles.


%%----------------------------------------------------
\subsubsection{Inferability}

%Participants talk about something they are interested in:
%it is often \fea{inferable} to some other things the participants or their interests are related to.
%The term \fea{inferable} is used in \citeA{prince81}
%to refer to a discourse referent
%that the NP representing ``is linked, by means of another NP, or ``Anchor'', [...] to some other discourse entity [discourse referent in this paper's term].''

The term \fea{inferable} is borrowed from \citeA{prince81}
though many other scholars have discussed similar concepts \cite[e.g.,][]{havilandclark74,chafe94}.
A discourse referent is inferable
``if the speaker assumes the hearer can infer it, via logical --  or, more commonly, plausible -- reasoning, from [discourse referents] already [active] or from other inferables'' \cite[][p.\ 236]{prince81}.%
	\footnote{
	The terms are replaced according to this study's terminology.
	}
A referent is inferable typically through
the part-whole or metonymic relationships between the referent and another referent that has been already active.
Inferable referents can be a topic
by being assumed to be shared between the speaker and the hearer,
or can be focus.

%%%To be modified (消防車の例)
%As has also been pointed out in Prince (xxxx),
%some entities can be easily inferred in a given culture,
%while others are not.


%%----------------------------------------------------
%\subsubsection{Entity vs.\ event}
%
%Topic is typically an entity,
%while focus is more frequently an event.
%This is related to the observation that
%predicate-focus structure, rather than argument-focus structure,
%is more common in the natural discourse \cite{lambrecht94}.
%In predicate-focus structure,
%the topic is an entity (e.g., \ci{Hanako} in \ci{Hanako read a book}),
%while the focus is an event (e.g., \ci{read a book}).
%Features such as % definite, specific, 
%animacy and agentivity are those of entities.
%Events cannot have these features.
%

%%----------------------------------------------------
%%----------------------------------------------------
\section{Methodology}\label{FrameworkTFIdent}

In this section,
I will discuss the methods in this study,
based on the definitions and assumptions of the topic and the focus specified in the last section.
This study employs acceptability judgements,
production experiments, and
corpus annotation,
to be discussed in the following sections.
%In acceptability judgements, production experiments, or corpus study,
%it is necessary to identify topics and foci with clearly defined criteria for each methodology.
%While it is possible to control contexts in acceptability judgements and production experiments,
%it is relatively harder to define them in naturally occurring data.

%%----------------------------------------------------
\subsection{Topic and focus in acceptability judgements}

In acceptability judgements,
I sometimes employ the \ci{hee} test,
where the element in question is focus if it can be repeated after the expression \ci{hee} `really',
while it is not if it cannot.
See also the discussions in \S \ref{BackSubsecDefTopic}, \ref{BackSubsecDefFocus}, \ref{FrameworkTopic}, and \ref{FrameworkFocus}.
The \ci{hee}-test is exemplified in \Next.%
 \footnote{
  \chd{
  Read Jiro's utterance in \ref{Fr:Method:Ex:Hebi} with exclamative intonation.
  Question intonation always works regardless of whether the element in question
  is topic or focus.
  }
 }
\ex.\label{Fr:Method:Ex:Hebi} \ag.[Taro:] kinoo-sa [ore]$_{T}$ [hebi mi-ta-n-da]$_{F}$-yo \\
		yesterday-\ab{fp} \ab{1}\ab{sg} snake see-\ab{past}-\ab{nmlz}-\ab{cop}-\ab{fp} \\
	`Yesterday [I]$_{T}$ [saw a snake]$_{F}$!'
	\bg.[Jiro:] hee, \{??kinoo / ??taroo / hebi (mi-ta-n-da)\}! \\
			really yesterday / Taro / snake (see-\ab{past}-\ab{nmlz}-\ab{cop}) \\
			`Really, yesterday? / you? / (saw) a snake?'

Let us assume that in \Last[-Taro] it is presupposed that something happened to Taro yesterday.
Since something must always happen to Taro all the time,
this presupposition is appropriate even in an out-of-the-blue context.
Therefore,
\ci{ore} `\ab{1}\ab{sg}' is interpreted as topic,
while \ci{hebi mi-ta-n-da} `snake see-\ab{past}-\ab{nmlz}-\ab{cop}' is interpreted as focus in this particular context.
Given this situation,
the hearer of \Last[-Taro] can respond to this utterance as in \Last[-Jiro]:
while the focus part \ci{hebi mi-ta-n-da} `snake see-\ab{past}-\ab{nmlz}-\ab{cop}' can be felicitously repeated followed by \ci{hee} `really',
the topic part \ci{ore} `\ab{1}\ab{sg}',
which corresponds to \ci{taroo} in \Last[-Jiro], cannot be repeated felicitously.
Topics are identified negatively in this test.
The assumption of this \ci{hee} test is that
topics can never be taken as ``news'' or ``a surprise'' since they are assumed to be shared between the speaker and the hearer,
while foci are expected to be ``news'' or ``a surprise'' to the hearer.

The expression \ci{kinoo} `yesterday' cannot be repeated either.
I assume that this is because \ci{kinoo} `yesterday' is also a part of presupposition.
However, I am neutral as to whether or not \ci{kinoo} `yesterday' is a topic
in the same sense that \ci{ore} `\ab{1}\ab{sg}' is a topic.
It is a kind of stage topic discussed in \ref{FrameworkIS}.
In this study I restrict myself to investigating elements which constitute arguments of sentences
and do not discuss much about the stage topics in detail.

In grammaticality judgements,
%If elements themselves are not clear as to whether they are topics or foci,
contexts will be provided
in order for topics to be typical topics (presupposed, definite, etc.)
and for foci to be typical foci (asserted, indefinite, etc.).
Examples of contexts which prompt different focus structures are provided in \ref{FrameworkExPredFocus} to \ref{FrameworkExArgFocus},
where the target expression is \ci{koinu(-o) yuzut-ta} `gave a/the puppy'.
%
	\ex. \label{FrameworkExPredFocus}
	\a.[] \EM{Predicate-focus context}: Yesterday the speaker and his/her friend found an abandoned puppy on the street. The speaker brought it to his/her home. Today, the speaker tells the friend what happened to the puppy.
	\bg.[A:] sooieba [\EM{koinu}]$_{T}$ [\EM{yuzut-ta}]$_{F}$-yo \\
		by.the.way puppy give-\ab{past}-\ab{fp} \\
		`By the way, (I) gave the puppy (to somebody).'
%	\hfill{\movie{play}{koinut.wav}}

		\ex. \label{FrameworkExSentFocus}
		\a.[] \EM{Sentence-focus context}: the speaker and his/her friend are working in an animal shelter. The friend was absent yesterday and wants to know what happened yesterday.
		\bg.[A:] kinoo-wa [\EM{koinu} \EM{yuzut-ta}]$_{F}$-yo \\
		yesterday-\ci{wa} puppy give-\ab{past}-\ab{fp} \\
		`Yesterday (we) gave a puppy.'
%	\hfill{\movie{play}{koinuf.wav}}

\ex. \label{FrameworkExArgFocus}\a.[] \EM{Argument-focus context}
	\b.[Q:] What did you give to him?
	\bg.[A:] [\EM{koinu-o}]$_{F}$ [\EM{yuzut-ta}]$_{T}$-yo \\
			 puppy-\ci{o} give-\ab{past}-\ab{fp} \\
			`(I) gave the/a puppy.'

In predicate-focus contexts like \ref{FrameworkExPredFocus},
typically the referent of the discourse element in question has already appeared in the context preceding the target expression;
in this example,
\ci{koinu} `puppy' has appeared in the context and the speaker and the hearer share the identity of the puppy.
Therefore,
\ci{koinu} `puppy' is easily presupposed and is interpreted as topic.
The speaker intends to tell the hearer what happened to the puppy
because this news is not shared with the hearer.
The readers may wonder
why I do not simply use a question like `what happened to the puppy?',
which typically prompts predicate-focus structure.
This question, however,
strongly favours omitting the element \ci{koinu} `puppy'
because it appears in the immediate context.
This is the reason why
the context which prompts predicate-focus structure like \ref{FrameworkExPredFocus} appears to be complicated.

In sentence-focus contexts like (\ref{FrameworkExSentFocus}),
on the other hand,
typically the referent is not shared;
in A of \ref{FrameworkExSentFocus},
\ci{koinu} `puppy' appears out-of-the-blue.
The whole utterance is interpreted as news or focus.
In this case, A of \ref{FrameworkExSentFocus} can be easily preceded by
questions like `what happened yesterday?'.

Argument-focus contexts like \ref{FrameworkExArgFocus}
are typically \ci{what}- or \ci{who}-questions
that prompt a single argument as answer.
In \ref{FrameworkExArgFocus},
the question prompts \ci{koinu} `puppy' as answer.
`A gave (something)' is presupposed.


%%----------------------------------------------------
\subsection{Assumptions in experiments}

In production experiments,
I asked Japanese native speakers to read aloud sentences preceded by different contexts:
the context where the sentence is interpreted as different types of focus structures.
%predicate-focus structure and where it is interpreted as sentence-focus structure.
The contexts that prompt different types of focus structures
are designed in the same way as discussed in the last section.

%%----------------------------------------------------
\subsection{Corpus annotation and analysis}\label{FrameworkCorpus}

In analyzing spontaneous speech,
it is relatively difficult to apply the definition of the topic and the focus discussed above
because clean contexts are not available in contrast to the case with constructed examples.
For this reason,
I will provide the definitions of topic and focus for the corpus investigation
based on the assumptions concerning topic and focus discussed in \S \ref{FrameworkDefinition}.
%Before providing the definitions,
The basic idea is that, since it is difficult to determine
whether some discourse referent is presupposed or not,
I will use information status to approximate the given-new taxonomy (\S \ref{FW:Cor:TopFoc})
%and the persistence (discussed below)
of the referent
%to identify a topic and a focus in the corpus
instead of the \fea{presupposed} vs.\ \fea{asserted} distinction.
The activation status of the referent in question is approximated
by whether the referent has an antecedent or not.
%The persistence of the referent is approximated by
%how many times the referent is mentioned in the following discourse \cite{givon83}.

Firstly, I will discuss the characteristics of the corpus (\S \ref{FW:Cor:Cor})
and the procedure of annotating anaphoric relations (\S \ref{FW:Cor:AnaRel}).
Then the annotations of relevant features will be discussed (\S \ref{FW:Cor:TopFoc}).
%Secondly, I will discuss how to identify topics and foci in the natural spoken corpus and the limitation of the identification process (\S \ref{FW:Cor:TopFoc}).
%Thirdly, the methods of annotating other features will be discussed (\S \ref{FW:Cor:OthFea}).
%Finally, the statistical methods employed in this study and their limitation will be discussed in \S \ref{FW:Cor:Stats}.

%%----------------------------------------------------
\subsubsection{Corpus}\label{FW:Cor:Cor}

This study investigates 12 core data of simulated public speaking 
from \ci{the Corpus of Spontaneous Japanese} \cite[CSJ:][]{maekawa03,maekawaetal04}.
The data list and basic information are summarized in Table \ref{CorpusInfoT}.
The data to be investigated are randomly chosen out of 107 core data of simulated public speaking.
Simulated public speaking is a type of speech
where the speakers talk about everyday topics
such as `my most delightful memory' or `if I live in a deserted island'.
I use the RDB version of CSJ \cite{koisoetal12} to search the corpus.

\begin{table}
	\begin{center}
	\caption{Corpus used in this study}
	\label{CorpusInfoT}
	\begin{tabular}{lllr}
	\toprule
	ID & Speaker gender (age) & Theme & Length (sec) \\
	\midrule
	S00F0014 & F (30-34) & Travel to Hawaii & 1269 \\
	S00F0209 & F (25-29) & Being a pianist & 619 \\
	S00M0199 & M (30-34) & Kosovo War & 580 \\
	S00M0221 & M (25-29) & Working at Sarakin & 654 \\
	S01F0038 & F (40-44) & Luck in getting jobs & 628 \\
	S01F0151 & F (30-34) & Trek in Himalayas & 765 \\
	S01M0182 & M (40-44) & Boxing & 644 \\
	S02M0198 & M (20-24) & Dog's death & 762 \\
	S02M1698 & M (65-69) & Dog's death & 649 \\
	S02F0100 & F (20-24) & Rare disease & 740 \\
	S03F0072 & F (35-39) & A year in Iran & 816 \\
	S05M1236 & M (30-34) & Memories in Mobara & 832 \\
	\bottomrule
	\end{tabular}
	\end{center}
\end{table}

The core data of CSJ has rich information of various kinds.
%\Next is the list of information that is relevant for this study.
I used the information in \Next to generate information relevant for this study.
%
\ex.
 \a. Utterance time
% \b. Morphological information
 \b. Dependency relation
 \b. Phrase \& clause boundary
 \b. Intonation

Relevant variables will be explained in each section.

%%----------------------------------------------------
\subsubsection{Annotation of anaphoric relations}\label{FW:Cor:AnaRel}

The information of anaphoric relations is used
to identify topic and focus.
Anaphoric relations are identified in the following way.
The basic procedures have been proposed in \citeA{iidaetal07} and \citeA{nakagawaden12}.
\ex.\label{AnnotationProcedure} 
	\a. \EM{Identification of grammatical function, discourse elements, and zero pronoun}
	\b. \EM{Classification of discourse elements}: Discourse elements are classified into categories based on what they refer to.
	\b. \EM{Identification of anaphoric relations}:
		The link between the anaphor and the antecedent is annotated.
%	\b. \EM{Annotation of topic continuity}:
%		The information structural features of each discourse element are computed
%		from the annotation of anaphoric relations.


First,
I identified the grammatical function of clauses (a in \ref{AnnotationProcedure}),
namely A, S, vs.~P.
This is necessary in order to determine discourse elements and zero pronouns to be investigated.
In Japanese,
pronouns such as \ci{watasi} `\ab{1}\ab{sg}', \ci{anata} `\ab{2}\ab{sg}', and \ci{kare} `\ab{3}\ab{sg}' are rare;
the most frequent pronoun is the zero pronoun.
In \Next,
for example, the speaker indicated by {\O$_{Sp}$} and `the dog' indicated by {\O$_{i}$} are zero pronouns,
assuming that they appear immediately before the predicates.%
	\
As shown in \Next[d],
two zero pronouns {\O$_{Sp}$} and {\O$_{i}$} can appear in the same clause;
still, native speakers have no trouble in understanding the utterance.
\ex. \ag. yo-nen-kan amerika-de sigoto-o {\O$_{Sp}$} si-teru aida \\
			four-year-for America-\ab{loc} work-\ci{o} {\O$_{Sp}$} do-\ab{prog} during \\
			`While (I) was working for four years,'
	\bg. aa zutto kono inu$_{i}$-to issyoni eii {\O$_{Sp}$} sun-de \\
		\ab{fl} all.the.time this dog-with together \ab{fl} {\O$_{Sp}$} live-and \\
		`(I) lived with this dog all the time.'
	\bg. sikamo oo tabi-o {\O$_{Sp}$} suru toki-mo \\
		moreover \ab{fl} travel-\ci{o} {\O$_{Sp}$} do time-also \\
		`Moreover, also when (I) travel,'
	\bg. kuruma-ni {\O$_{Sp}$} {\O$_{i}$} nose-te \\
		car-\ab{loc} {\O$_{Sp}$} {\O$_{i}$} put-and \\
		`(I) put (the dog) in my car.'
	\bg. ee amerika-o tabi {\O$_{Sp}$} si-ta-to \\
		\ab{fl} America-\ab{acc} travel {\O$_{Sp}$} do-\ab{past}-\ab{q} \\
		`(I) traveled America.'
		\hfill{\code{(S02M1698: 182.88-195.87)}}
%四年間アメリカで仕事をしてる間
%あーずっとこの犬と一緒にえいー住んで
%しかもおー旅をする時も
%車に乗せて
%えーアメリカを旅したと
%(S02M1698: 182.88-195.87)

I identified 7697 discourse elements (5234 NPs, 655 overt pronouns, and 1808 zero pronouns) from the corpus.

Second, I classified discourse elements into 13 categories depending on what they refer to b in \ref{AnnotationProcedure}:
common referent, connective, speaker, hearer, time, filler, exophora, question, quantifier, degree words, proposition, and other.
Although there are many categories,
only common referents are relevant for the purpose of this study.
Other categories were annotated for future studies.
Also, I limit my analyses to A, S, P, and Ex (to be discussed below).
Datives are also added for comparison.
This process leaves us
2301 elements (1662 NPs, 80 overt pronouns, and 559 zero pronouns).
However, I occasionally use data which include other kinds of elements
for detailed analysis.

Third,
I identified the anaphoric relation for each discourse element (c in \ref{AnnotationProcedure}).
A unique ID number is given for the set of discourse elements which refer to the same entity.
In \Next, for example,
\ci{syoo-doobutu} `a small animal' in line a,
\ci{\O} in line c, e, and f refers to the small animal introduced in line a.
All of them are given the ID number 1 because they refer to the same entity.
The element \ci{syoo-doobutu} `a small animal' is called the \EM{antecedent} of the \EM{anaphor} \ci{\O} in line c.
In the same way, the element \ci{\O} in line c is the antecedent of the \EM{anaphor} \ci{\O} in line e.
The element \ci{watasi} refers to another entity, the speaker,
and is given another ID number 2.
\ex.
		\begin{tabular}{llr}
		 & & ID \\
		\rowcolor{gray}
		a. & \sstack{\EM{syoo-doobutu}-ga koo tyokotyoko-to \\ ki-ta-n-desu-ne} & 1  \\
		\rowcolor{gray}
		 & `\EM{A small animal} came (towards us) with small steps.' & \\
		b. & de saisyo koo  & -- \\
		 & `and at first, so...' & \\
		\rowcolor{gray}
		c. & \sstack{ano sotira-no soto-no-hoo-kara \\ \EM{\O} nozoi-ta-mon-desu-kara} & 1 \\
		\rowcolor{gray}
		 & `uh \EM{it} looked at us from that direction, so' & \\
		d. & \ul{watasi}-wa saisyo & 2 \\
		 & `At first, I...' & \\
		\rowcolor{gray}
		e. & \EM{\O} risu-kana-to omot-ta-n-desu & 1 \\
		\rowcolor{gray}
		 & `(I) thoguht that \EM{it} was a squirrel.' & \\
%		f. & de & -- & -- \\
%		 & `and' & & \\
%		g. & t= sat-to koo & -- & -- \\
%		 & `quickly' & & \\
%		h. & are-to omot-te it-tara & -- & -- \\
%		 & `when I was thinking something,' & & \\
		f. & [...] sat-to \EM{\O} nige-tyai-masi-te & 1 \\
		 & `\EM{it} quickly ran away, and' & \\
		\end{tabular}
		\begin{flushright}
		(\code{S00F0014: 619.51-631.71})
		\end{flushright}


Using the anaphoric relations and various information in the corpus,
I generated other relevant features to be discussed in the next section.

%%----------------------------------------------------
\subsubsection{Annotation of topichood and focushood}\label{FW:Cor:TopFoc}

%%----------------------------------------------------
\paragraph{Approximation to the given-new taxonomy}

The status of a referent in the given-new taxonomy is approximated by
whether the expression referring to the referent has an antecedent or not.
An expression that has an antecedent is called an \EM{anaphoric} element,
while an expression that does not have an antecedent is called a \EM{non-anaphoric} element.
I use the term information status to refer to the status of a referent being anaphoric or non-anaphoric.
\chd{Note that the terms anaphoric vs.~non-anaphoric are used in Chapter \ref{Particles}, \ref{WordOrder}, and \ref{Intonation} only to refer to corpus counts.}
The referent of an anaphoric elements is assumed to be either evoked or declining in terms of the given-new taxonomy and active or semi-active in terms of activation status.
On the other hand, the referent of a non-anaphoric elements is inferable, unused, or new in terms of the given-new taxonomy and
semi-active or inactive in terms of activation status.
I prefer to use the terms of the given-new taxonomy over activation status because
they are more fine-grained.
The correspondence among activation statuses, the given-new taxonomy, and corpus annotations are shown in Table \ref{ActStatusCorpus}.
The distinction between inferable, declining, unused, and brand-new is judged manually when necessary.
By ``shared'', I mean the referent is evoked, declining, inferable, or unused in terms of the given-new taxonomy.

\begin{table}
	\caption{Activation status, the given-new taxonomy, and corpus annotation}
	\label{ActStatusCorpus}
	\begin{center}
	\begin{tabular}{llll}
	\toprule
	Activation status & \multicolumn{2}{l}{The given-new taxonomy} & Corpus annotation \\
	\midrule
	Active & Evoked & & Anaphoric \\
	Semi-active & Declining &  & \\
	\rowcolor{gray}
	Semi-active & Inferable &  & \\
	\rowcolor{gray}
	Inactive & Unused & \rdelim\}{-4}{20pt}[Shared] & Non-anaphoric \\
	\rowcolor{gray}
	Inactive & Brand-new & &  \\
	\bottomrule
	\end{tabular}\\
	\end{center}
\end{table}


%%----------------------------------------------------
\paragraph{Grammatical function}

Following \citeA{comrie78} and \citeA{dixon79},
I distinguish S, A, and P in grammatical function.
S is the only argument of intransitive clause,
A is the agent-like argument of transitive clause,
and P is the patient-like argument of transitive clause.
For now, I simply distinguish A and P based on whether the argument in question is or can be coded by \ci{ga} or \ci{o}.
When it can be coded by \ci{ga}, it is A;
when it can be coded by \ci{o}, it is P.
Furthermore, I sometimes distinguish agent S and patient S if needed.
%since the distinction plays an important role in the spoken Japanese grammar.
%For now, I had to rely on the intuition to distinguish agent and patient S.
%I employ the criteria proposed in the literature to avoid unreliability.
%The criteria will be explained in the relevant chapters.

In addition to S, A, and P,
I identify non-argument elements (Ex).\label{FW:Cor:TopFoc:ExDef}
Non-argument elements are those which appear to be part of the clause but do not have direct relationships with the predicate.
A typical example is shown in \Next.
%
\exg. \EM{zoo-wa} hana-ga nagai \\
		elephant-\ci{wa} nose-\ci{ga} long \\
		`The elephant, the nose is long (The elephant has a long nose).' \hfill{\cite{mikami60}}

As exemplified in \Last,
the element \ci{zoo} `elephant' is considered to be Ex.
\ci{Hana} `nose' is the only argument of the predicate (S),
and \ci{zoo} `elephant' does not have direct relationships with the predicate \ci{nagai} `long';
still, \ci{zoo} `elephant' looks like part of the clause and needs a label,
which happens to be ``Ex''.

Although Ex is frequently coded by so-called topic markers
such as \ci{wa} and \ci{toiuno-wa},
\ci{wa}- and \ci{toiuno-wa}-coded elements are not always labelled as Ex.
If they are considered to be S, A, or P,
they are labelled as such.
For example, in the case where \ci{hana} `nose' is coded by \ci{wa} like \Next,
\ci{nose} is labelled as S, instead of Ex.
%
\exg. zoo-no hana-wa nagai \\
      elephant-\ab{gen} nose-\ci{wa} long \\
      `The elephant's nose is long.'






%In \Last,
%the element coded by \ci{wa}, \ci{toiuno-wa}, or \ci{mo}
%are often non-arguments,
%and hence this feature needs to be annotated.


%%----------------------------------------------------
\paragraph{Other features}

Ideally, it is necessary to annotate all the variables proposed in \ref{ISFeatures},
but it is impossible to annotate all of them,
partially because of the limitation of time and labor and 
partially because of the lack of clear criteria to annotate them consistently.
For example, definiteness and specificity are difficult to annotate consistently.
Multiple annotators are needed for reliable and objective analyses.
Animacy could be simpler, but I have not annotated this feature throughout the corpus due to the limitation of time and labor.
The previous literature indicates that
these features play little role in Japanese grammar.
These features will be discussed when necessary.

%In investigating a spoken corpus,
%topic and focus are identified by the activation status.
% and the persistence.
%As discussed in \S \ref{FrameworkActivation},
%activated referents are more likely to be topics.
%On the other hand, inactivated referents are more likely to be focus.
%As \citeA{lambrecht94} and many others have pointed out,
%activated referents are more acceptable topics than inactivated referents.
%The topic acceptability scale is summarized as in \Next,
%where I paraphrased Lambrecht's terms with this study's terminology in parentheses.
%\ex. \label{TopicAcceptabilityScale}The topic acceptability scale
%	\a.[$\uparrow$] \tl{most acceptable topic}
%	\b. active (activated)
%	\b. accessible (semi-activated/inferable)
%	\b. unused (inactivated and definite or generic)
%	\b. brand-new anchored (inactivated and maybe definite or indefinite)
%	\b. brand-new unanchored (inactivated and indefinite)
%	\b.[$\downarrow$] \tl{least acceptable topic}
%		\hfill{\cite[][p.\ 165]{lambrecht94}}
%
%Table \ref{ActStatus} is a summary of \Last and my argument so far:
%activated referents are typical, or most frequent, topics, and
%new (inactivated) referents are typical foci.
%I will keep the term ``unused'' in \Last for convenience and, henceforth,
%I will exclude unused referents from inactivated referents.
%
%\begin{table}
%	\caption{Activation status and topic/focus}
%	\label{ActStatus}
%	\begin{center}
%	\begin{tabular}{lcc}
%	\toprule
%	& Topic & Focus \\
%	\midrule
%	Activated & \cellcolor{gray}{Typical topic} & \\
%	Inferable &  & \\
%	Semi-activated &  & \\
%	Unused & & \\
%	Inactivated &  & \cellcolor{gray}{Typical focus} \\
%	\bottomrule
%	\end{tabular}\\
%	\end{center}
%\end{table}
%
%In my corpus investigation,
%the activation status of a referent is approximated by
%whether the element referring to the referent in question has an antecedent or not.
%I will call this approximation \EM{information status}.
%If the element has an antecedent,
%it indicates that the referent is activated
%because it has already been mentioned by the antecedent.
%If an element has no antecedent,
%it is more likely to be inactivated in the hear's mind.
%This method is employed in annotations of information structure
%\cite{givon83,calhounetal05,gotzeetal07}
%.
%Note, however, that
%the referent of an element without an antecedent are not always inactivated.
%The referent may be activated through the visual perception,
%or it may be partially activated through the part-whole relations.
%To keep in mind that the correlation between the actual activation status and having an antecedent is not perfect,
%I will refer to the referent of an element that has antecedent as \EM{given} referent,
%and the referent of an element without an antecedent as \EM{new} referent.
%The relation between the activation status and given vs.\ new distinction is summarized in Table \ref{ActStatusCorpus}.
%I assume that given referents are more likely to be topics and new referents are more likely to be foci in the corpus investigation.
%The referents that are inferable, semi-activated, or unused will be analyzed qualitatively followed by quantitative investigations.
%
%\begin{table}
%	\caption{Activation status in the corpus}
%	\label{ActStatusCorpus}
%	\begin{center}
%	\begin{tabular}{lcc}
%	\toprule
%	& Topic & Focus \\
%	\midrule
%	Activated & \multicolumn{2}{c}{Given in the corpus} \\
%	\rowcolor{gray}
%	Inferable &  & \\
%	\rowcolor{gray}
%	Semi-activated &  & \\
%	\rowcolor{gray}
%	Unused & \multicolumn{2}{c}{New in the corpus} \\
%	\rowcolor{gray}
%	Inactivated &  & \\
%	\bottomrule
%	\end{tabular}\\
%	\end{center}
%\end{table}
%
%I will employ another diagnostics of topics and foci:
%persistence \cite{givon83}.
%A persistent referent is a referent that is mentioned more than once throughout the discourse,
%while a non-persistent referent is a referent that is mentioned only once throughout the discourse.
%Note that I modified the criteria of persistence from Giv\'on's original ones.
%The domain where the criteria can be applied is each speech that the speaker gave, which is simply separated by files.
%I assume that persistent referents are more likely to be topics,
%while non-persistent referents are more likely to be foci
%following \cite{givon83}.

%Table \ref{TFhierarchyT}

%\begin{table}
%	\begin{center}
%	\caption{Activation status and other features of non-generic referent}
%	\label{ISFeaturesST}
%%	\small
%	\begin{tabular}{lccccc}
%	\toprule
%	& Activated & \sstack{Can be topic \\ (or presupposed)} & \sstack{Can be Focus \\ (or asserted)} & definite & Specific \\
%	\midrule
%	\EM{Activated} & \Circle & \Circle & \Circle & \Circle & \Circle \\
%	\EM{Semi-activated} & \RIGHTcircle & \Circle & \Circle & \Circle & \Circle \\
%	\EM{Inferable} & \RIGHTcircle & \Circle & \Circle & \RIGHTcircle & \RIGHTcircle \\
%	\EM{Unused} & \CIRCLE & \Circle & \Circle & \Circle & \Circle \\
%%	\EM{Unmentioned} & \CIRCLE & \Circle & \Circle & \RIGHTcircle & \CIRCLE \\
%	\EM{Brand new} & \CIRCLE & \CIRCLE & \Circle & \CIRCLE & \RIGHTcircle \\
%	\bottomrule
%	\end{tabular}\\
%%	\hfill{(features in higher position override those in lower position)}
%	\end{center}
%\end{table}
%
%\begin{table}
%	\begin{center}
%	\caption{Activation status and other features of generic referent}
%	\label{ISFeaturesGT}
%%	\small
%	\begin{tabular}{lccccc}
%	\toprule
%	& Activated & \sstack{Can be topic \\ (or presupposed)} & \sstack{Can be Focus \\ (or asserted)} & definite & Specific \\
%	\midrule
%	\EM{Activated} & \Circle & \Circle & \Circle & \RIGHTcircle & \CIRCLE \\
%	\EM{Semi-activated} & \RIGHTcircle & \Circle & \Circle & \RIGHTcircle & \RIGHTcircle \\
%	\EM{Inferable} & \RIGHTcircle & \Circle & \Circle & \RIGHTcircle & \RIGHTcircle \\
%%	\EM{Unused} & \CIRCLE & \Circle & \Circle & \Circle & \Circle \\
%	\EM{Unmentioned} & \CIRCLE & \Circle & \Circle & \CIRCLE & \CIRCLE \\
%%	\EM{Brand new} & \CIRCLE & \CIRCLE & \Circle & \CIRCLE & \RIGHTcircle \\
%	\bottomrule
%	\end{tabular}\\
%%	\hfill{(features in higher position override those in lower position)}
%	\end{center}
%\end{table}


%Information structure features \\ \cite[see also]{prince81,chafe94}
%\begin{table}
%	\begin{center}
%	\caption{Topic-focus distinction and given-new hyerarchy}
%	\label{TFhierarchyT}
%%	\small
%	\begin{tabular}{lcc}
%	\toprule
%	& Topic & Focus \\
%	\midrule
%	\EM{Activated} & \Circle & \Circle \\
%	\EM{Semi-activated} & \Circle &  \Circle \\
%	\EM{Inferable} & \Circle &  \Circle  \\
%	\EM{Unused} & \Circle & \Circle \\
%%	\EM{Unmentioned} & \Circle & \Circle \\
%	\EM{Brand new} & \CIRCLE & \Circle \\
%	\bottomrule
%	\end{tabular}\\
%%	\hfill{(features in higher position override those in lower position)}
%	\end{center}
%\end{table}

%%----------------------------------------------------
%\subsubsection{Annotation of other features}\label{FW:Cor:OthFea}




%%%----------------------------------------------------
%\subsubsection{Notes on statistics}\label{FW:Cor:Stats}
%
%Ideally, all the features relating to information structure should be annotated, and
%a regression analysis should be applied to investigate
%which feature contributes to which part of spoken Japanese grammar.
%However, it turned out that
%my data is still too small to apply a regression analysis;
%%therefore, I mainly employ Fischer's exact test.
%I sometimes apply statistical tests when appropriate.
%The tool used for statistical analyses and graphics throughout the study
%is R.%
%	\footnote{
%	\href{http://www.r-project.org/}{http://www.r-project.org/}
%	}


%%----------------------------------------------------
\section{Summary}

In this chapter,
I discussed
the framework employed in this study and
the method of corpus annotation and analysis.
In the next three chapters,
different aspects of spoken Japanese grammar (i.e., particles, word order, and intonation) will be analyzed based on the framework and methodology discussed in this chapter.








